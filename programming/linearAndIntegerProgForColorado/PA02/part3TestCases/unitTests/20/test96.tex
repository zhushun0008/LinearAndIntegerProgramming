\documentclass[9pt]{article}
\usepackage{amsmath}
\begin{document}
Read in the following dictionary:
\[\begin{array}{c| c c@{\hskip 2pt} c@{\hskip 2pt} c@{\hskip 2pt} c@{\hskip 2pt} c@{\hskip 2pt} c@{\hskip 2pt} c@{\hskip 2pt} c@{\hskip 2pt} c@{\hskip 2pt} c@{\hskip 2pt} c@{\hskip 2pt} c@{\hskip 2pt} c@{\hskip 2pt} c@{\hskip 2pt} c@{\hskip 2pt} c@{\hskip 2pt} c@{\hskip 2pt} }
 x_{18}   &  3.0 & +  6.00 x_{1} & +  6.00 x_{2} & +  8.00 x_{3} & -2.00 x_{4} & -1.00 x_{5} & -3.00 x_{6} &   & -2.00 x_{8} & -3.00 x_{9} & -8.00 x_{10} & +  6.00 x_{11} & -7.00 x_{12} & -4.00 x_{13} & +  2.00 x_{14} & +  2.00 x_{15} & -5.00 x_{16} & -8.00 x_{17}\\
 x_{19}   &  -1.0 & +  9.00 x_{1} & +  7.00 x_{2} & -7.00 x_{3} & +  9.00 x_{4} & +  3.00 x_{5} &   & +  8.00 x_{7} & -2.00 x_{8} & -7.00 x_{9} & +  3.00 x_{10} & +  6.00 x_{11} & +  6.00 x_{12} & +  5.00 x_{13} & -3.00 x_{14} & +  8.00 x_{15} & +  3.00 x_{16} & -5.00 x_{17}\\
 x_{20}   &  -3.0 & -4.00 x_{1} & -3.00 x_{2} & -7.00 x_{3} & +  8.00 x_{4} &   & +  6.00 x_{6} & + 10.00 x_{7} & +  1.00 x_{8} & -3.00 x_{9} & +  5.00 x_{10} & -3.00 x_{11} & +  1.00 x_{12} & +  8.00 x_{13} & +  1.00 x_{14} & +  9.00 x_{15} & +  8.00 x_{16} & -6.00 x_{17}\\
 x_{21}   &  3.0 & -3.00 x_{1} &    &   & +  1.00 x_{4} & -10.00 x_{5} & +  2.00 x_{6} & +  1.00 x_{7} & +  2.00 x_{8} & +  8.00 x_{9} & -8.00 x_{10} & +  8.00 x_{11} & -9.00 x_{12} & +  4.00 x_{13} &   & +  5.00 x_{15} & -9.00 x_{16} & -9.00 x_{17}\\
 x_{22}   &  1.0 & +  7.00 x_{1} & +  2.00 x_{2} & -6.00 x_{3} &   & +  8.00 x_{5} & +  2.00 x_{6} & -1.00 x_{7} & +  2.00 x_{8} & +  7.00 x_{9} & +  2.00 x_{10} & +  6.00 x_{11} & -9.00 x_{12} & +  7.00 x_{13} & +  2.00 x_{14} & +  2.00 x_{15} & -5.00 x_{16} & +  6.00 x_{17}\\
 x_{23}   &  1.0 & +  3.00 x_{1} & +  5.00 x_{2} & -10.00 x_{3} & -3.00 x_{4} & -8.00 x_{5} & -8.00 x_{6} & +  9.00 x_{7} & -4.00 x_{8} & +  2.00 x_{9} & -2.00 x_{10} & +  1.00 x_{11} & -5.00 x_{12} & -10.00 x_{13} & +  9.00 x_{14} & + 10.00 x_{15} & -5.00 x_{16} & -7.00 x_{17}\\
 x_{24}   &  3.0 & -9.00 x_{1} & -3.00 x_{2} & -3.00 x_{3} & +  1.00 x_{4} & -9.00 x_{5} & -5.00 x_{6} & -5.00 x_{7} & -9.00 x_{8} & +  6.00 x_{9} & +  3.00 x_{10} &   & -3.00 x_{12} &   & -3.00 x_{14} & + 10.00 x_{15} & +  8.00 x_{16} & -5.00 x_{17}\\
 x_{25}   &  -3.0 & +  4.00 x_{1} & + 10.00 x_{2} & +  2.00 x_{3} & -2.00 x_{4} & +  4.00 x_{5} & +  9.00 x_{6} & -2.00 x_{7} & -3.00 x_{8} & -5.00 x_{9} & -10.00 x_{10} & -4.00 x_{11} & -6.00 x_{12} &   & -6.00 x_{14} & +  6.00 x_{15} & -9.00 x_{16} & +  8.00 x_{17}\\
 x_{26}   &  -3.0 & + 10.00 x_{1} & -4.00 x_{2} & + 10.00 x_{3} & +  5.00 x_{4} & +  9.00 x_{5} & -8.00 x_{6} & +  9.00 x_{7} & +  7.00 x_{8} & -9.00 x_{9} & + 10.00 x_{10} & +  2.00 x_{11} & +  6.00 x_{12} & -1.00 x_{13} & -4.00 x_{14} & +  8.00 x_{15} & -5.00 x_{16} & -3.00 x_{17}\\
 x_{27}   &  0.0 & +  4.00 x_{1} & +  7.00 x_{2} & +  6.00 x_{3} &   & +  1.00 x_{5} & +  3.00 x_{6} & +  6.00 x_{7} & +  4.00 x_{8} & +  9.00 x_{9} & -3.00 x_{10} & +  7.00 x_{11} & + 10.00 x_{12} & +  9.00 x_{13} & -7.00 x_{14} & -8.00 x_{15} & -2.00 x_{16} & -5.00 x_{17}\\
 x_{28}   &  -2.0 & -4.00 x_{1} & +  8.00 x_{2} & -4.00 x_{3} & +  9.00 x_{4} & -8.00 x_{5} & -5.00 x_{6} & +  1.00 x_{7} & +  4.00 x_{8} & -5.00 x_{9} & -1.00 x_{10} & +  9.00 x_{11} & -9.00 x_{12} & +  7.00 x_{13} & +  7.00 x_{14} & +  8.00 x_{15} & +  2.00 x_{16} & +  7.00 x_{17}\\
 x_{29}   &  0.0 & +  8.00 x_{1} & -4.00 x_{2} & -8.00 x_{3} & -4.00 x_{4} & +  3.00 x_{5} & +  3.00 x_{6} & +  2.00 x_{7} & +  8.00 x_{8} & -4.00 x_{9} & -2.00 x_{10} &   & -4.00 x_{12} & +  6.00 x_{13} & -3.00 x_{14} & +  6.00 x_{15} & -6.00 x_{16} & +  2.00 x_{17}\\
 x_{30}   &  3.0 & +  5.00 x_{1} & +  7.00 x_{2} & -8.00 x_{3} & -4.00 x_{4} & +  3.00 x_{5} & -3.00 x_{6} & -6.00 x_{7} & +  4.00 x_{8} & -5.00 x_{9} & +  6.00 x_{10} & -6.00 x_{11} & -2.00 x_{12} & -7.00 x_{13} & -9.00 x_{14} & + 10.00 x_{15} & -9.00 x_{16} & +  3.00 x_{17}\\
 x_{31}   &  3.0  &   & -2.00 x_{2} & -3.00 x_{3} & +  4.00 x_{4} & -7.00 x_{5} & +  3.00 x_{6} & +  2.00 x_{7} & +  5.00 x_{8} & +  1.00 x_{9} & +  8.00 x_{10} & -2.00 x_{11} & +  9.00 x_{12} & +  4.00 x_{13} & +  5.00 x_{14} & +  9.00 x_{15} & -10.00 x_{16} & -5.00 x_{17}\\
 x_{32}   &  -1.0 & -4.00 x_{1} & +  1.00 x_{2} & + 10.00 x_{3} & + 10.00 x_{4} & +  1.00 x_{5} & +  9.00 x_{6} & +  1.00 x_{7} & +  7.00 x_{8} & + 10.00 x_{9} & +  6.00 x_{10} & -10.00 x_{11} & +  5.00 x_{12} & -5.00 x_{13} & +  8.00 x_{14} & -9.00 x_{15} & -4.00 x_{16} &   \\
\hline
z    &  0.0 & +  3.00 x_{1} & +  1.00 x_{2} &   & +  3.00 x_{4} & -1.00 x_{5} & -2.00 x_{6} & -3.00 x_{7} & +  5.00 x_{8} & -1.00 x_{9} & +  1.00 x_{10} & +  4.00 x_{11} & +  1.00 x_{12} & +  1.00 x_{13} & +  5.00 x_{14} & +  3.00 x_{15} & +  2.00 x_{16} & -2.00 x_{17}\\
\end{array}\]
\subsection{Initialization Phase: Dual Problem Solving}
New Objective in primal was changed to : \[ \max\ \sum_{j=1}^{17}\ - x_j \] 
Primal variable $x_j$ corresponds to dual variable $y_j$ for $j = 1,\ldots,32$
Dual Dictionary (with objective changed is): 
\[\begin{array}{c| c c@{\hskip 2pt} c@{\hskip 2pt} c@{\hskip 2pt} c@{\hskip 2pt} c@{\hskip 2pt} c@{\hskip 2pt} c@{\hskip 2pt} c@{\hskip 2pt} c@{\hskip 2pt} c@{\hskip 2pt} c@{\hskip 2pt} c@{\hskip 2pt} c@{\hskip 2pt} c@{\hskip 2pt} c@{\hskip 2pt} }
 y_{1}   &  1.0 & -6.00 y_{18} & -9.00 y_{19} & +  4.00 y_{20} & +  3.00 y_{21} & -7.00 y_{22} & -3.00 y_{23} & +  9.00 y_{24} & -4.00 y_{25} & -10.00 y_{26} & -4.00 y_{27} & +  4.00 y_{28} & -8.00 y_{29} & -5.00 y_{30} &   & +  4.00 y_{32}\\
 y_{2}   &  1.0 & -6.00 y_{18} & -7.00 y_{19} & +  3.00 y_{20} &   & -2.00 y_{22} & -5.00 y_{23} & +  3.00 y_{24} & -10.00 y_{25} & +  4.00 y_{26} & -7.00 y_{27} & -8.00 y_{28} & +  4.00 y_{29} & -7.00 y_{30} & +  2.00 y_{31} & -1.00 y_{32}\\
 y_{3}   &  1.0 & -8.00 y_{18} & +  7.00 y_{19} & +  7.00 y_{20} &   & +  6.00 y_{22} & + 10.00 y_{23} & +  3.00 y_{24} & -2.00 y_{25} & -10.00 y_{26} & -6.00 y_{27} & +  4.00 y_{28} & +  8.00 y_{29} & +  8.00 y_{30} & +  3.00 y_{31} & -10.00 y_{32}\\
 y_{4}   &  1.0 & +  2.00 y_{18} & -9.00 y_{19} & -8.00 y_{20} & -1.00 y_{21} &   & +  3.00 y_{23} & -1.00 y_{24} & +  2.00 y_{25} & -5.00 y_{26} &   & -9.00 y_{28} & +  4.00 y_{29} & +  4.00 y_{30} & -4.00 y_{31} & -10.00 y_{32}\\
 y_{5}   &  1.0 & +  1.00 y_{18} & -3.00 y_{19} &   & + 10.00 y_{21} & -8.00 y_{22} & +  8.00 y_{23} & +  9.00 y_{24} & -4.00 y_{25} & -9.00 y_{26} & -1.00 y_{27} & +  8.00 y_{28} & -3.00 y_{29} & -3.00 y_{30} & +  7.00 y_{31} & -1.00 y_{32}\\
 y_{6}   &  1.0 & +  3.00 y_{18} &   & -6.00 y_{20} & -2.00 y_{21} & -2.00 y_{22} & +  8.00 y_{23} & +  5.00 y_{24} & -9.00 y_{25} & +  8.00 y_{26} & -3.00 y_{27} & +  5.00 y_{28} & -3.00 y_{29} & +  3.00 y_{30} & -3.00 y_{31} & -9.00 y_{32}\\
 y_{7}   &  1.0  &   & -8.00 y_{19} & -10.00 y_{20} & -1.00 y_{21} & +  1.00 y_{22} & -9.00 y_{23} & +  5.00 y_{24} & +  2.00 y_{25} & -9.00 y_{26} & -6.00 y_{27} & -1.00 y_{28} & -2.00 y_{29} & +  6.00 y_{30} & -2.00 y_{31} & -1.00 y_{32}\\
 y_{8}   &  1.0 & +  2.00 y_{18} & +  2.00 y_{19} & -1.00 y_{20} & -2.00 y_{21} & -2.00 y_{22} & +  4.00 y_{23} & +  9.00 y_{24} & +  3.00 y_{25} & -7.00 y_{26} & -4.00 y_{27} & -4.00 y_{28} & -8.00 y_{29} & -4.00 y_{30} & -5.00 y_{31} & -7.00 y_{32}\\
 y_{9}   &  1.0 & +  3.00 y_{18} & +  7.00 y_{19} & +  3.00 y_{20} & -8.00 y_{21} & -7.00 y_{22} & -2.00 y_{23} & -6.00 y_{24} & +  5.00 y_{25} & +  9.00 y_{26} & -9.00 y_{27} & +  5.00 y_{28} & +  4.00 y_{29} & +  5.00 y_{30} & -1.00 y_{31} & -10.00 y_{32}\\
 y_{10}   &  1.0 & +  8.00 y_{18} & -3.00 y_{19} & -5.00 y_{20} & +  8.00 y_{21} & -2.00 y_{22} & +  2.00 y_{23} & -3.00 y_{24} & + 10.00 y_{25} & -10.00 y_{26} & +  3.00 y_{27} & +  1.00 y_{28} & +  2.00 y_{29} & -6.00 y_{30} & -8.00 y_{31} & -6.00 y_{32}\\
 y_{11}   &  1.0 & -6.00 y_{18} & -6.00 y_{19} & +  3.00 y_{20} & -8.00 y_{21} & -6.00 y_{22} & -1.00 y_{23} &   & +  4.00 y_{25} & -2.00 y_{26} & -7.00 y_{27} & -9.00 y_{28} &   & +  6.00 y_{30} & +  2.00 y_{31} & + 10.00 y_{32}\\
 y_{12}   &  1.0 & +  7.00 y_{18} & -6.00 y_{19} & -1.00 y_{20} & +  9.00 y_{21} & +  9.00 y_{22} & +  5.00 y_{23} & +  3.00 y_{24} & +  6.00 y_{25} & -6.00 y_{26} & -10.00 y_{27} & +  9.00 y_{28} & +  4.00 y_{29} & +  2.00 y_{30} & -9.00 y_{31} & -5.00 y_{32}\\
 y_{13}   &  1.0 & +  4.00 y_{18} & -5.00 y_{19} & -8.00 y_{20} & -4.00 y_{21} & -7.00 y_{22} & + 10.00 y_{23} &    &   & +  1.00 y_{26} & -9.00 y_{27} & -7.00 y_{28} & -6.00 y_{29} & +  7.00 y_{30} & -4.00 y_{31} & +  5.00 y_{32}\\
 y_{14}   &  1.0 & -2.00 y_{18} & +  3.00 y_{19} & -1.00 y_{20} &   & -2.00 y_{22} & -9.00 y_{23} & +  3.00 y_{24} & +  6.00 y_{25} & +  4.00 y_{26} & +  7.00 y_{27} & -7.00 y_{28} & +  3.00 y_{29} & +  9.00 y_{30} & -5.00 y_{31} & -8.00 y_{32}\\
 y_{15}   &  1.0 & -2.00 y_{18} & -8.00 y_{19} & -9.00 y_{20} & -5.00 y_{21} & -2.00 y_{22} & -10.00 y_{23} & -10.00 y_{24} & -6.00 y_{25} & -8.00 y_{26} & +  8.00 y_{27} & -8.00 y_{28} & -6.00 y_{29} & -10.00 y_{30} & -9.00 y_{31} & +  9.00 y_{32}\\
 y_{16}   &  1.0 & +  5.00 y_{18} & -3.00 y_{19} & -8.00 y_{20} & +  9.00 y_{21} & +  5.00 y_{22} & +  5.00 y_{23} & -8.00 y_{24} & +  9.00 y_{25} & +  5.00 y_{26} & +  2.00 y_{27} & -2.00 y_{28} & +  6.00 y_{29} & +  9.00 y_{30} & + 10.00 y_{31} & +  4.00 y_{32}\\
 y_{17}   &  1.0 & +  8.00 y_{18} & +  5.00 y_{19} & +  6.00 y_{20} & +  9.00 y_{21} & -6.00 y_{22} & +  7.00 y_{23} & +  5.00 y_{24} & -8.00 y_{25} & +  3.00 y_{26} & +  5.00 y_{27} & -7.00 y_{28} & -2.00 y_{29} & -3.00 y_{30} & +  5.00 y_{31} &   \\
\hline
z    &  -0 & -3.00 y_{18} & +  1.00 y_{19} & +  3.00 y_{20} & -3.00 y_{21} & -1.00 y_{22} & -1.00 y_{23} & -3.00 y_{24} & +  3.00 y_{25} & +  3.00 y_{26} &   & +  2.00 y_{28} &   & -3.00 y_{30} & -3.00 y_{31} & +  1.00 y_{32}\\
\end{array}\]
Initialization succeeded in finding final dual dictionary with 16 pivots
\[\begin{array}{c| c c@{\hskip 2pt} c@{\hskip 2pt} c@{\hskip 2pt} c@{\hskip 2pt} c@{\hskip 2pt} c@{\hskip 2pt} c@{\hskip 2pt} c@{\hskip 2pt} c@{\hskip 2pt} c@{\hskip 2pt} c@{\hskip 2pt} c@{\hskip 2pt} c@{\hskip 2pt} c@{\hskip 2pt} c@{\hskip 2pt} }
 y_{5}   &  0.146128887947 & +  5.68 y_{18} & +  0.24 y_{7} & -9.57 y_{19} & + 12.47 y_{21} & -11.73 y_{22} & +  4.63 y_{23} & +  8.79 y_{24} & +  0.42 y_{15} & +  0.72 y_{3} & +  0.31 y_{6} & -0.33 y_{2} & -1.43 y_{29} & -7.17 y_{30} & +  8.69 y_{31} & -0.51 y_{4}\\
 y_{26}   &  0.0601289999198 & -0.42 y_{18} & -0.02 y_{7} & +  0.10 y_{19} & -0.13 y_{21} & +  0.29 y_{22} & -0.06 y_{23} & -0.20 y_{24} & -0.03 y_{15} & -0.05 y_{3} & +  0.02 y_{6} & +  0.01 y_{2} & +  0.14 y_{29} & +  0.18 y_{30} & -0.12 y_{31} & +  0.01 y_{4}\\
 y_{28}   &  0.0182669446437 & -0.11 y_{18} & +  0.02 y_{7} & -0.73 y_{19} & +  0.05 y_{21} & -0.14 y_{22} & -0.40 y_{23} & -0.21 y_{24} & +  0.00 y_{15} & +  0.02 y_{3} & +  0.04 y_{6} & -0.04 y_{2} & +  0.36 y_{29} & -0.45 y_{30} & -0.01 y_{31} & -0.05 y_{4}\\
 y_{1}   &  0.289704349004 & +  2.35 y_{18} & +  0.40 y_{7} & -12.90 y_{19} & +  4.20 y_{21} & -10.63 y_{22} & +  1.17 y_{23} & +  8.09 y_{24} & +  0.35 y_{15} & +  0.65 y_{3} & -0.12 y_{6} & +  0.12 y_{2} & -8.47 y_{29} & -5.23 y_{30} & -1.38 y_{31} & -0.68 y_{4}\\
 y_{32}   &  0.0394167396623 & +  0.09 y_{18} & +  0.03 y_{7} & +  0.17 y_{19} & -0.01 y_{21} & +  0.12 y_{22} & +  1.05 y_{23} & +  0.04 y_{24} & +  0.01 y_{15} & -0.02 y_{3} & -0.02 y_{6} & +  0.02 y_{2} & +  0.39 y_{29} & +  0.49 y_{30} & -0.07 y_{31} & -0.05 y_{4}\\
 y_{20}   &  0.0419747296911 & +  0.43 y_{18} & -0.04 y_{7} & -0.57 y_{19} & -0.15 y_{21} & -0.18 y_{22} & -0.42 y_{23} & +  0.15 y_{24} & -0.01 y_{15} & +  0.03 y_{3} & -0.03 y_{6} & +  0.02 y_{2} & -0.50 y_{29} & +  0.18 y_{30} & -0.40 y_{31} & -0.01 y_{4}\\
 y_{27}   &  0.0294034591747 & -0.21 y_{18} & -0.07 y_{7} & -0.43 y_{19} & +  0.19 y_{21} & +  0.02 y_{22} & -0.78 y_{23} & +  0.85 y_{24} & +  0.04 y_{15} & +  0.01 y_{3} & +  0.01 y_{6} & -0.05 y_{2} & +  0.14 y_{29} & +  0.28 y_{30} & +  0.43 y_{31} & +  0.03 y_{4}\\
 y_{8}   &  0.363042581276 & +  3.99 y_{18} & +  0.21 y_{7} & +  5.46 y_{19} & -2.58 y_{21} & -4.35 y_{22} & +  2.53 y_{23} & +  6.93 y_{24} & -0.16 y_{15} & +  0.26 y_{3} & -0.23 y_{6} & +  0.09 y_{2} & -13.53 y_{29} & -9.51 y_{30} & -5.76 y_{31} & +  0.47 y_{4}\\
 y_{9}   &  1.58715322123 & -1.13 y_{18} & +  0.22 y_{7} & +  4.90 y_{19} & -12.29 y_{21} & -7.27 y_{22} & -8.70 y_{23} & -17.19 y_{24} & -1.00 y_{15} & -0.24 y_{3} & +  0.14 y_{6} & +  0.10 y_{2} & -0.32 y_{29} & -4.83 y_{30} & -7.83 y_{31} & +  0.19 y_{4}\\
 y_{10}   &  1.03387691426 & +  4.71 y_{18} & +  0.14 y_{7} & -3.80 y_{19} & +  8.09 y_{21} & -5.30 y_{22} & -3.23 y_{23} & -1.29 y_{24} & -0.05 y_{15} & +  0.21 y_{3} & -0.17 y_{6} & -0.72 y_{2} & +  0.40 y_{29} & -15.78 y_{30} & -5.69 y_{31} & +  0.55 y_{4}\\
 y_{11}   &  1.41963234093 & -2.12 y_{18} & +  0.66 y_{7} & +  3.47 y_{19} & -11.09 y_{21} & -4.97 y_{22} & + 17.81 y_{23} & -3.43 y_{24} & -0.34 y_{15} & -0.43 y_{3} & -0.92 y_{6} & +  0.81 y_{2} & -2.51 y_{29} & + 11.32 y_{30} & -3.63 y_{31} & -0.19 y_{4}\\
 y_{12}   &  0.855573689155 & +  7.31 y_{18} & +  0.94 y_{7} & -8.89 y_{19} & +  6.98 y_{21} & +  5.03 y_{22} & +  5.40 y_{23} & -7.55 y_{24} & -0.53 y_{15} & +  0.26 y_{3} & +  0.01 y_{6} & -0.22 y_{2} & +  2.80 y_{29} & -11.32 y_{30} & -13.50 y_{31} & -0.32 y_{4}\\
 y_{13}   &  0.528915115624 & +  3.29 y_{18} & +  1.04 y_{7} & +  9.42 y_{19} & -5.01 y_{21} & -3.82 y_{22} & + 28.37 y_{23} & -7.36 y_{24} & -0.30 y_{15} & -0.65 y_{3} & -0.22 y_{6} & +  0.70 y_{2} & -3.62 y_{29} & +  8.85 y_{30} & -5.15 y_{31} & -0.10 y_{4}\\
 y_{14}   &  1.54620116104 & -7.99 y_{18} & -0.80 y_{7} & +  4.92 y_{19} & -0.92 y_{21} & -0.75 y_{22} & -19.20 y_{23} & +  8.26 y_{24} & -0.20 y_{15} & -0.27 y_{3} & -0.11 y_{6} & -0.27 y_{2} & -1.31 y_{29} & +  8.04 y_{30} & -3.00 y_{31} & +  1.10 y_{4}\\
 y_{25}   &  0.0975063677717 & -0.41 y_{18} & +  0.02 y_{7} & +  0.04 y_{19} & -0.27 y_{21} & -0.05 y_{22} & +  0.11 y_{23} & -0.16 y_{24} & -0.05 y_{15} & -0.03 y_{3} & -0.04 y_{6} & -0.02 y_{2} & -0.12 y_{29} & -0.46 y_{30} & -0.26 y_{31} & +  0.02 y_{4}\\
 y_{16}   &  2.02234445973 & -4.09 y_{18} & +  0.37 y_{7} & +  3.59 y_{19} & +  7.37 y_{21} & +  8.29 y_{22} & + 12.51 y_{23} & -9.34 y_{24} & -0.38 y_{15} & -0.87 y_{3} & -0.16 y_{6} & -0.23 y_{2} & + 10.78 y_{29} & +  7.77 y_{30} & + 10.85 y_{31} & +  0.25 y_{4}\\
 y_{17}   &  0.6713331191 & + 12.32 y_{18} & -0.91 y_{7} & +  4.56 y_{19} & + 10.42 y_{21} & -4.71 y_{22} & +  2.30 y_{23} & + 12.35 y_{24} & +  0.42 y_{15} & +  0.20 y_{3} & -0.01 y_{6} & +  0.38 y_{2} & -5.45 y_{29} & +  6.84 y_{30} & +  6.55 y_{31} & +  0.24 y_{4}\\
\hline
z    &  0.674780921098 & -4.35 y_{18} & -0.07 y_{7} & -1.59 y_{19} & -4.56 y_{21} & -0.99 y_{22} & -1.86 y_{23} & -4.00 y_{24} & -0.24 y_{15} & -0.13 y_{3} & -0.11 y_{6} & -0.06 y_{2} & -0.36 y_{29} & -3.73 y_{30} & -5.41 y_{31} & -0.07 y_{4}\\
\end{array}\]
Primal Dictionary is:
\[\begin{array}{c| c c@{\hskip 2pt} c@{\hskip 2pt} c@{\hskip 2pt} c@{\hskip 2pt} c@{\hskip 2pt} c@{\hskip 2pt} c@{\hskip 2pt} c@{\hskip 2pt} c@{\hskip 2pt} c@{\hskip 2pt} c@{\hskip 2pt} c@{\hskip 2pt} c@{\hskip 2pt} c@{\hskip 2pt} c@{\hskip 2pt} c@{\hskip 2pt} c@{\hskip 2pt} }
 x_{18}   &  4.35437402255 & -5.68 x_{5} & +  0.42 x_{26} & +  0.11 x_{28} & -2.35 x_{1} & -0.09 x_{32} & -0.43 x_{20} & +  0.21 x_{27} & -3.99 x_{8} & +  1.13 x_{9} & -4.71 x_{10} & +  2.12 x_{11} & -7.31 x_{12} & -3.29 x_{13} & +  7.99 x_{14} & +  0.41 x_{25} & +  4.09 x_{16} & -12.32 x_{17}\\
 x_{7}   &  0.0663403086752 & -0.24 x_{5} & +  0.02 x_{26} & -0.02 x_{28} & -0.40 x_{1} & -0.03 x_{32} & +  0.04 x_{20} & +  0.07 x_{27} & -0.21 x_{8} & -0.22 x_{9} & -0.14 x_{10} & -0.66 x_{11} & -0.94 x_{12} & -1.04 x_{13} & +  0.80 x_{14} & -0.02 x_{25} & -0.37 x_{16} & +  0.91 x_{17}\\
 x_{19}   &  1.59233905241 & +  9.57 x_{5} & -0.10 x_{26} & +  0.73 x_{28} & + 12.90 x_{1} & -0.17 x_{32} & +  0.57 x_{20} & +  0.43 x_{27} & -5.46 x_{8} & -4.90 x_{9} & +  3.80 x_{10} & -3.47 x_{11} & +  8.89 x_{12} & -9.42 x_{13} & -4.92 x_{14} & -0.04 x_{25} & -3.59 x_{16} & -4.56 x_{17}\\
 x_{21}   &  4.55567423615 & -12.47 x_{5} & +  0.13 x_{26} & -0.05 x_{28} & -4.20 x_{1} & +  0.01 x_{32} & +  0.15 x_{20} & -0.19 x_{27} & +  2.58 x_{8} & + 12.29 x_{9} & -8.09 x_{10} & + 11.09 x_{11} & -6.98 x_{12} & +  5.01 x_{13} & +  0.92 x_{14} & +  0.27 x_{25} & -7.37 x_{16} & -10.42 x_{17}\\
 x_{22}   &  0.986151226264 & + 11.73 x_{5} & -0.29 x_{26} & +  0.14 x_{28} & + 10.63 x_{1} & -0.12 x_{32} & +  0.18 x_{20} & -0.02 x_{27} & +  4.35 x_{8} & +  7.27 x_{9} & +  5.30 x_{10} & +  4.97 x_{11} & -5.03 x_{12} & +  3.82 x_{13} & +  0.75 x_{14} & +  0.05 x_{25} & -8.29 x_{16} & +  4.71 x_{17}\\
 x_{23}   &  1.86341667086 & -4.63 x_{5} & +  0.06 x_{26} & +  0.40 x_{28} & -1.17 x_{1} & -1.05 x_{32} & +  0.42 x_{20} & +  0.78 x_{27} & -2.53 x_{8} & +  8.70 x_{9} & +  3.23 x_{10} & -17.81 x_{11} & -5.40 x_{12} & -28.37 x_{13} & + 19.20 x_{14} & -0.11 x_{25} & -12.51 x_{16} & -2.30 x_{17}\\
 x_{24}   &  3.9988253429 & -8.79 x_{5} & +  0.20 x_{26} & +  0.21 x_{28} & -8.09 x_{1} & -0.04 x_{32} & -0.15 x_{20} & -0.85 x_{27} & -6.93 x_{8} & + 17.19 x_{9} & +  1.29 x_{10} & +  3.43 x_{11} & +  7.55 x_{12} & +  7.36 x_{13} & -8.26 x_{14} & +  0.16 x_{25} & +  9.34 x_{16} & -12.35 x_{17}\\
 x_{15}   &  0.238017681459 & -0.42 x_{5} & +  0.03 x_{26} & -0.00 x_{28} & -0.35 x_{1} & -0.01 x_{32} & +  0.01 x_{20} & -0.04 x_{27} & +  0.16 x_{8} & +  1.00 x_{9} & +  0.05 x_{10} & +  0.34 x_{11} & +  0.53 x_{12} & +  0.30 x_{13} & +  0.20 x_{14} & +  0.05 x_{25} & +  0.38 x_{16} & -0.42 x_{17}\\
 x_{3}   &  0.127505492 & -0.72 x_{5} & +  0.05 x_{26} & -0.02 x_{28} & -0.65 x_{1} & +  0.02 x_{32} & -0.03 x_{20} & -0.01 x_{27} & -0.26 x_{8} & +  0.24 x_{9} & -0.21 x_{10} & +  0.43 x_{11} & -0.26 x_{12} & +  0.65 x_{13} & +  0.27 x_{14} & +  0.03 x_{25} & +  0.87 x_{16} & -0.20 x_{17}\\
 x_{6}   &  0.113536321616 & -0.31 x_{5} & -0.02 x_{26} & -0.04 x_{28} & +  0.12 x_{1} & +  0.02 x_{32} & +  0.03 x_{20} & -0.01 x_{27} & +  0.23 x_{8} & -0.14 x_{9} & +  0.17 x_{10} & +  0.92 x_{11} & -0.01 x_{12} & +  0.22 x_{13} & +  0.11 x_{14} & +  0.04 x_{25} & +  0.16 x_{16} & +  0.01 x_{17}\\
 x_{2}   &  0.057208240396 & +  0.33 x_{5} & -0.01 x_{26} & +  0.04 x_{28} & -0.12 x_{1} & -0.02 x_{32} & -0.02 x_{20} & +  0.05 x_{27} & -0.09 x_{8} & -0.10 x_{9} & +  0.72 x_{10} & -0.81 x_{11} & +  0.22 x_{12} & -0.70 x_{13} & +  0.27 x_{14} & +  0.02 x_{25} & +  0.23 x_{16} & -0.38 x_{17}\\
 x_{29}   &  0.363827265559 & +  1.43 x_{5} & -0.14 x_{26} & -0.36 x_{28} & +  8.47 x_{1} & -0.39 x_{32} & +  0.50 x_{20} & -0.14 x_{27} & + 13.53 x_{8} & +  0.32 x_{9} & -0.40 x_{10} & +  2.51 x_{11} & -2.80 x_{12} & +  3.62 x_{13} & +  1.31 x_{14} & +  0.12 x_{25} & -10.78 x_{16} & +  5.45 x_{17}\\
 x_{30}   &  3.73324823665 & +  7.17 x_{5} & -0.18 x_{26} & +  0.45 x_{28} & +  5.23 x_{1} & -0.49 x_{32} & -0.18 x_{20} & -0.28 x_{27} & +  9.51 x_{8} & +  4.83 x_{9} & + 15.78 x_{10} & -11.32 x_{11} & + 11.32 x_{12} & -8.85 x_{13} & -8.04 x_{14} & +  0.46 x_{25} & -7.77 x_{16} & -6.84 x_{17}\\
 x_{31}   &  5.40720726634 & -8.69 x_{5} & +  0.12 x_{26} & +  0.01 x_{28} & +  1.38 x_{1} & +  0.07 x_{32} & +  0.40 x_{20} & -0.43 x_{27} & +  5.76 x_{8} & +  7.83 x_{9} & +  5.69 x_{10} & +  3.63 x_{11} & + 13.50 x_{12} & +  5.15 x_{13} & +  3.00 x_{14} & +  0.26 x_{25} & -10.85 x_{16} & -6.55 x_{17}\\
 x_{4}   &  0.0721728769516 & +  0.51 x_{5} & -0.01 x_{26} & +  0.05 x_{28} & +  0.68 x_{1} & +  0.05 x_{32} & +  0.01 x_{20} & -0.03 x_{27} & -0.47 x_{8} & -0.19 x_{9} & -0.55 x_{10} & +  0.19 x_{11} & +  0.32 x_{12} & +  0.10 x_{13} & -1.10 x_{14} & -0.02 x_{25} & -0.25 x_{16} & -0.24 x_{17}\\
\hline
z    &  -0.674780921098 & -0.15 x_{5} & -0.06 x_{26} & -0.02 x_{28} & -0.29 x_{1} & -0.04 x_{32} & -0.04 x_{20} & -0.03 x_{27} & -0.36 x_{8} & -1.59 x_{9} & -1.03 x_{10} & -1.42 x_{11} & -0.86 x_{12} & -0.53 x_{13} & -1.55 x_{14} & -0.10 x_{25} & -2.02 x_{16} & -0.67 x_{17}\\
\end{array}\]
Primal Dictionary with original objective is:
\[\begin{array}{c| c c@{\hskip 2pt} c@{\hskip 2pt} c@{\hskip 2pt} c@{\hskip 2pt} c@{\hskip 2pt} c@{\hskip 2pt} c@{\hskip 2pt} c@{\hskip 2pt} c@{\hskip 2pt} c@{\hskip 2pt} c@{\hskip 2pt} c@{\hskip 2pt} c@{\hskip 2pt} c@{\hskip 2pt} c@{\hskip 2pt} c@{\hskip 2pt} c@{\hskip 2pt} }
 x_{18}   &  4.35437402255 & -5.68 x_{5} & +  0.42 x_{26} & +  0.11 x_{28} & -2.35 x_{1} & -0.09 x_{32} & -0.43 x_{20} & +  0.21 x_{27} & -3.99 x_{8} & +  1.13 x_{9} & -4.71 x_{10} & +  2.12 x_{11} & -7.31 x_{12} & -3.29 x_{13} & +  7.99 x_{14} & +  0.41 x_{25} & +  4.09 x_{16} & -12.32 x_{17}\\
 x_{7}   &  0.0663403086752 & -0.24 x_{5} & +  0.02 x_{26} & -0.02 x_{28} & -0.40 x_{1} & -0.03 x_{32} & +  0.04 x_{20} & +  0.07 x_{27} & -0.21 x_{8} & -0.22 x_{9} & -0.14 x_{10} & -0.66 x_{11} & -0.94 x_{12} & -1.04 x_{13} & +  0.80 x_{14} & -0.02 x_{25} & -0.37 x_{16} & +  0.91 x_{17}\\
 x_{19}   &  1.59233905241 & +  9.57 x_{5} & -0.10 x_{26} & +  0.73 x_{28} & + 12.90 x_{1} & -0.17 x_{32} & +  0.57 x_{20} & +  0.43 x_{27} & -5.46 x_{8} & -4.90 x_{9} & +  3.80 x_{10} & -3.47 x_{11} & +  8.89 x_{12} & -9.42 x_{13} & -4.92 x_{14} & -0.04 x_{25} & -3.59 x_{16} & -4.56 x_{17}\\
 x_{21}   &  4.55567423615 & -12.47 x_{5} & +  0.13 x_{26} & -0.05 x_{28} & -4.20 x_{1} & +  0.01 x_{32} & +  0.15 x_{20} & -0.19 x_{27} & +  2.58 x_{8} & + 12.29 x_{9} & -8.09 x_{10} & + 11.09 x_{11} & -6.98 x_{12} & +  5.01 x_{13} & +  0.92 x_{14} & +  0.27 x_{25} & -7.37 x_{16} & -10.42 x_{17}\\
 x_{22}   &  0.986151226264 & + 11.73 x_{5} & -0.29 x_{26} & +  0.14 x_{28} & + 10.63 x_{1} & -0.12 x_{32} & +  0.18 x_{20} & -0.02 x_{27} & +  4.35 x_{8} & +  7.27 x_{9} & +  5.30 x_{10} & +  4.97 x_{11} & -5.03 x_{12} & +  3.82 x_{13} & +  0.75 x_{14} & +  0.05 x_{25} & -8.29 x_{16} & +  4.71 x_{17}\\
 x_{23}   &  1.86341667086 & -4.63 x_{5} & +  0.06 x_{26} & +  0.40 x_{28} & -1.17 x_{1} & -1.05 x_{32} & +  0.42 x_{20} & +  0.78 x_{27} & -2.53 x_{8} & +  8.70 x_{9} & +  3.23 x_{10} & -17.81 x_{11} & -5.40 x_{12} & -28.37 x_{13} & + 19.20 x_{14} & -0.11 x_{25} & -12.51 x_{16} & -2.30 x_{17}\\
 x_{24}   &  3.9988253429 & -8.79 x_{5} & +  0.20 x_{26} & +  0.21 x_{28} & -8.09 x_{1} & -0.04 x_{32} & -0.15 x_{20} & -0.85 x_{27} & -6.93 x_{8} & + 17.19 x_{9} & +  1.29 x_{10} & +  3.43 x_{11} & +  7.55 x_{12} & +  7.36 x_{13} & -8.26 x_{14} & +  0.16 x_{25} & +  9.34 x_{16} & -12.35 x_{17}\\
 x_{15}   &  0.238017681459 & -0.42 x_{5} & +  0.03 x_{26} & -0.00 x_{28} & -0.35 x_{1} & -0.01 x_{32} & +  0.01 x_{20} & -0.04 x_{27} & +  0.16 x_{8} & +  1.00 x_{9} & +  0.05 x_{10} & +  0.34 x_{11} & +  0.53 x_{12} & +  0.30 x_{13} & +  0.20 x_{14} & +  0.05 x_{25} & +  0.38 x_{16} & -0.42 x_{17}\\
 x_{3}   &  0.127505492 & -0.72 x_{5} & +  0.05 x_{26} & -0.02 x_{28} & -0.65 x_{1} & +  0.02 x_{32} & -0.03 x_{20} & -0.01 x_{27} & -0.26 x_{8} & +  0.24 x_{9} & -0.21 x_{10} & +  0.43 x_{11} & -0.26 x_{12} & +  0.65 x_{13} & +  0.27 x_{14} & +  0.03 x_{25} & +  0.87 x_{16} & -0.20 x_{17}\\
 x_{6}   &  0.113536321616 & -0.31 x_{5} & -0.02 x_{26} & -0.04 x_{28} & +  0.12 x_{1} & +  0.02 x_{32} & +  0.03 x_{20} & -0.01 x_{27} & +  0.23 x_{8} & -0.14 x_{9} & +  0.17 x_{10} & +  0.92 x_{11} & -0.01 x_{12} & +  0.22 x_{13} & +  0.11 x_{14} & +  0.04 x_{25} & +  0.16 x_{16} & +  0.01 x_{17}\\
 x_{2}   &  0.057208240396 & +  0.33 x_{5} & -0.01 x_{26} & +  0.04 x_{28} & -0.12 x_{1} & -0.02 x_{32} & -0.02 x_{20} & +  0.05 x_{27} & -0.09 x_{8} & -0.10 x_{9} & +  0.72 x_{10} & -0.81 x_{11} & +  0.22 x_{12} & -0.70 x_{13} & +  0.27 x_{14} & +  0.02 x_{25} & +  0.23 x_{16} & -0.38 x_{17}\\
 x_{29}   &  0.363827265559 & +  1.43 x_{5} & -0.14 x_{26} & -0.36 x_{28} & +  8.47 x_{1} & -0.39 x_{32} & +  0.50 x_{20} & -0.14 x_{27} & + 13.53 x_{8} & +  0.32 x_{9} & -0.40 x_{10} & +  2.51 x_{11} & -2.80 x_{12} & +  3.62 x_{13} & +  1.31 x_{14} & +  0.12 x_{25} & -10.78 x_{16} & +  5.45 x_{17}\\
 x_{30}   &  3.73324823665 & +  7.17 x_{5} & -0.18 x_{26} & +  0.45 x_{28} & +  5.23 x_{1} & -0.49 x_{32} & -0.18 x_{20} & -0.28 x_{27} & +  9.51 x_{8} & +  4.83 x_{9} & + 15.78 x_{10} & -11.32 x_{11} & + 11.32 x_{12} & -8.85 x_{13} & -8.04 x_{14} & +  0.46 x_{25} & -7.77 x_{16} & -6.84 x_{17}\\
 x_{31}   &  5.40720726634 & -8.69 x_{5} & +  0.12 x_{26} & +  0.01 x_{28} & +  1.38 x_{1} & +  0.07 x_{32} & +  0.40 x_{20} & -0.43 x_{27} & +  5.76 x_{8} & +  7.83 x_{9} & +  5.69 x_{10} & +  3.63 x_{11} & + 13.50 x_{12} & +  5.15 x_{13} & +  3.00 x_{14} & +  0.26 x_{25} & -10.85 x_{16} & -6.55 x_{17}\\
 x_{4}   &  0.0721728769516 & +  0.51 x_{5} & -0.01 x_{26} & +  0.05 x_{28} & +  0.68 x_{1} & +  0.05 x_{32} & +  0.01 x_{20} & -0.03 x_{27} & -0.47 x_{8} & -0.19 x_{9} & -0.55 x_{10} & +  0.19 x_{11} & +  0.32 x_{12} & +  0.10 x_{13} & -1.10 x_{14} & -0.02 x_{25} & -0.25 x_{16} & -0.24 x_{17}\\
\hline
z    &  0.561686346371 & +  0.93 x_{5} & +  0.05 x_{26} & +  0.30 x_{28} & +  4.84 x_{1} & +  0.15 x_{32} & -0.17 x_{20} & -0.35 x_{27} & +  4.14 x_{8} & +  2.29 x_{9} & +  0.29 x_{10} & +  4.92 x_{11} & +  6.63 x_{12} & +  4.19 x_{13} & -0.05 x_{14} & +  0.06 x_{25} & +  3.41 x_{16} & -7.15 x_{17}\\
\end{array}\]
\section{Optimization Phase Simplex}
Starting Dictionary is:
\[\begin{array}{c| c c@{\hskip 2pt} c@{\hskip 2pt} c@{\hskip 2pt} c@{\hskip 2pt} c@{\hskip 2pt} c@{\hskip 2pt} c@{\hskip 2pt} c@{\hskip 2pt} c@{\hskip 2pt} c@{\hskip 2pt} c@{\hskip 2pt} c@{\hskip 2pt} c@{\hskip 2pt} c@{\hskip 2pt} c@{\hskip 2pt} c@{\hskip 2pt} c@{\hskip 2pt} }
 x_{18}   &  4.35437402255 & -5.68 x_{5} & +  0.42 x_{26} & +  0.11 x_{28} & -2.35 x_{1} & -0.09 x_{32} & -0.43 x_{20} & +  0.21 x_{27} & -3.99 x_{8} & +  1.13 x_{9} & -4.71 x_{10} & +  2.12 x_{11} & -7.31 x_{12} & -3.29 x_{13} & +  7.99 x_{14} & +  0.41 x_{25} & +  4.09 x_{16} & -12.32 x_{17}\\
 x_{7}   &  0.0663403086752 & -0.24 x_{5} & +  0.02 x_{26} & -0.02 x_{28} & -0.40 x_{1} & -0.03 x_{32} & +  0.04 x_{20} & +  0.07 x_{27} & -0.21 x_{8} & -0.22 x_{9} & -0.14 x_{10} & -0.66 x_{11} & -0.94 x_{12} & -1.04 x_{13} & +  0.80 x_{14} & -0.02 x_{25} & -0.37 x_{16} & +  0.91 x_{17}\\
 x_{19}   &  1.59233905241 & +  9.57 x_{5} & -0.10 x_{26} & +  0.73 x_{28} & + 12.90 x_{1} & -0.17 x_{32} & +  0.57 x_{20} & +  0.43 x_{27} & -5.46 x_{8} & -4.90 x_{9} & +  3.80 x_{10} & -3.47 x_{11} & +  8.89 x_{12} & -9.42 x_{13} & -4.92 x_{14} & -0.04 x_{25} & -3.59 x_{16} & -4.56 x_{17}\\
 x_{21}   &  4.55567423615 & -12.47 x_{5} & +  0.13 x_{26} & -0.05 x_{28} & -4.20 x_{1} & +  0.01 x_{32} & +  0.15 x_{20} & -0.19 x_{27} & +  2.58 x_{8} & + 12.29 x_{9} & -8.09 x_{10} & + 11.09 x_{11} & -6.98 x_{12} & +  5.01 x_{13} & +  0.92 x_{14} & +  0.27 x_{25} & -7.37 x_{16} & -10.42 x_{17}\\
 x_{22}   &  0.986151226264 & + 11.73 x_{5} & -0.29 x_{26} & +  0.14 x_{28} & + 10.63 x_{1} & -0.12 x_{32} & +  0.18 x_{20} & -0.02 x_{27} & +  4.35 x_{8} & +  7.27 x_{9} & +  5.30 x_{10} & +  4.97 x_{11} & -5.03 x_{12} & +  3.82 x_{13} & +  0.75 x_{14} & +  0.05 x_{25} & -8.29 x_{16} & +  4.71 x_{17}\\
 x_{23}   &  1.86341667086 & -4.63 x_{5} & +  0.06 x_{26} & +  0.40 x_{28} & -1.17 x_{1} & -1.05 x_{32} & +  0.42 x_{20} & +  0.78 x_{27} & -2.53 x_{8} & +  8.70 x_{9} & +  3.23 x_{10} & -17.81 x_{11} & -5.40 x_{12} & -28.37 x_{13} & + 19.20 x_{14} & -0.11 x_{25} & -12.51 x_{16} & -2.30 x_{17}\\
 x_{24}   &  3.9988253429 & -8.79 x_{5} & +  0.20 x_{26} & +  0.21 x_{28} & -8.09 x_{1} & -0.04 x_{32} & -0.15 x_{20} & -0.85 x_{27} & -6.93 x_{8} & + 17.19 x_{9} & +  1.29 x_{10} & +  3.43 x_{11} & +  7.55 x_{12} & +  7.36 x_{13} & -8.26 x_{14} & +  0.16 x_{25} & +  9.34 x_{16} & -12.35 x_{17}\\
 x_{15}   &  0.238017681459 & -0.42 x_{5} & +  0.03 x_{26} & -0.00 x_{28} & -0.35 x_{1} & -0.01 x_{32} & +  0.01 x_{20} & -0.04 x_{27} & +  0.16 x_{8} & +  1.00 x_{9} & +  0.05 x_{10} & +  0.34 x_{11} & +  0.53 x_{12} & +  0.30 x_{13} & +  0.20 x_{14} & +  0.05 x_{25} & +  0.38 x_{16} & -0.42 x_{17}\\
 x_{3}   &  0.127505492 & -0.72 x_{5} & +  0.05 x_{26} & -0.02 x_{28} & -0.65 x_{1} & +  0.02 x_{32} & -0.03 x_{20} & -0.01 x_{27} & -0.26 x_{8} & +  0.24 x_{9} & -0.21 x_{10} & +  0.43 x_{11} & -0.26 x_{12} & +  0.65 x_{13} & +  0.27 x_{14} & +  0.03 x_{25} & +  0.87 x_{16} & -0.20 x_{17}\\
 x_{6}   &  0.113536321616 & -0.31 x_{5} & -0.02 x_{26} & -0.04 x_{28} & +  0.12 x_{1} & +  0.02 x_{32} & +  0.03 x_{20} & -0.01 x_{27} & +  0.23 x_{8} & -0.14 x_{9} & +  0.17 x_{10} & +  0.92 x_{11} & -0.01 x_{12} & +  0.22 x_{13} & +  0.11 x_{14} & +  0.04 x_{25} & +  0.16 x_{16} & +  0.01 x_{17}\\
 x_{2}   &  0.057208240396 & +  0.33 x_{5} & -0.01 x_{26} & +  0.04 x_{28} & -0.12 x_{1} & -0.02 x_{32} & -0.02 x_{20} & +  0.05 x_{27} & -0.09 x_{8} & -0.10 x_{9} & +  0.72 x_{10} & -0.81 x_{11} & +  0.22 x_{12} & -0.70 x_{13} & +  0.27 x_{14} & +  0.02 x_{25} & +  0.23 x_{16} & -0.38 x_{17}\\
 x_{29}   &  0.363827265559 & +  1.43 x_{5} & -0.14 x_{26} & -0.36 x_{28} & +  8.47 x_{1} & -0.39 x_{32} & +  0.50 x_{20} & -0.14 x_{27} & + 13.53 x_{8} & +  0.32 x_{9} & -0.40 x_{10} & +  2.51 x_{11} & -2.80 x_{12} & +  3.62 x_{13} & +  1.31 x_{14} & +  0.12 x_{25} & -10.78 x_{16} & +  5.45 x_{17}\\
 x_{30}   &  3.73324823665 & +  7.17 x_{5} & -0.18 x_{26} & +  0.45 x_{28} & +  5.23 x_{1} & -0.49 x_{32} & -0.18 x_{20} & -0.28 x_{27} & +  9.51 x_{8} & +  4.83 x_{9} & + 15.78 x_{10} & -11.32 x_{11} & + 11.32 x_{12} & -8.85 x_{13} & -8.04 x_{14} & +  0.46 x_{25} & -7.77 x_{16} & -6.84 x_{17}\\
 x_{31}   &  5.40720726634 & -8.69 x_{5} & +  0.12 x_{26} & +  0.01 x_{28} & +  1.38 x_{1} & +  0.07 x_{32} & +  0.40 x_{20} & -0.43 x_{27} & +  5.76 x_{8} & +  7.83 x_{9} & +  5.69 x_{10} & +  3.63 x_{11} & + 13.50 x_{12} & +  5.15 x_{13} & +  3.00 x_{14} & +  0.26 x_{25} & -10.85 x_{16} & -6.55 x_{17}\\
 x_{4}   &  0.0721728769516 & +  0.51 x_{5} & -0.01 x_{26} & +  0.05 x_{28} & +  0.68 x_{1} & +  0.05 x_{32} & +  0.01 x_{20} & -0.03 x_{27} & -0.47 x_{8} & -0.19 x_{9} & -0.55 x_{10} & +  0.19 x_{11} & +  0.32 x_{12} & +  0.10 x_{13} & -1.10 x_{14} & -0.02 x_{25} & -0.25 x_{16} & -0.24 x_{17}\\
\hline
z    &  0.561686346371 & +  0.93 x_{5} & +  0.05 x_{26} & +  0.30 x_{28} & +  4.84 x_{1} & +  0.15 x_{32} & -0.17 x_{20} & -0.35 x_{27} & +  4.14 x_{8} & +  2.29 x_{9} & +  0.29 x_{10} & +  4.92 x_{11} & +  6.63 x_{12} & +  4.19 x_{13} & -0.05 x_{14} & +  0.06 x_{25} & +  3.41 x_{16} & -7.15 x_{17}\\
\end{array}\]


 $ x_{1} $ enters and $ x_{7} $ leaves 

 \[\begin{array}{c| c c@{\hskip 2pt} c@{\hskip 2pt} c@{\hskip 2pt} c@{\hskip 2pt} c@{\hskip 2pt} c@{\hskip 2pt} c@{\hskip 2pt} c@{\hskip 2pt} c@{\hskip 2pt} c@{\hskip 2pt} c@{\hskip 2pt} c@{\hskip 2pt} c@{\hskip 2pt} c@{\hskip 2pt} c@{\hskip 2pt} c@{\hskip 2pt} c@{\hskip 2pt} }
 x_{18}   &  3.96470757081 & -4.25 x_{5} & +  0.33 x_{26} & +  0.20 x_{28} & +  5.87 x_{7} & +  0.10 x_{32} & -0.69 x_{20} & -0.22 x_{27} & -2.77 x_{8} & +  2.44 x_{9} & -3.88 x_{10} & +  5.97 x_{11} & -1.76 x_{12} & +  2.83 x_{13} & +  3.30 x_{14} & +  0.51 x_{25} & +  6.25 x_{16} & -17.68 x_{17}\\
 x_{1}   &  0.165900575348 & -0.60 x_{5} & +  0.04 x_{26} & -0.04 x_{28} & -2.50 x_{7} & -0.08 x_{32} & +  0.11 x_{20} & +  0.18 x_{27} & -0.52 x_{8} & -0.56 x_{9} & -0.35 x_{10} & -1.64 x_{11} & -2.36 x_{12} & -2.60 x_{13} & +  2.00 x_{14} & -0.04 x_{25} & -0.92 x_{16} & +  2.28 x_{17}\\
 x_{19}   &  3.73189217753 & +  1.77 x_{5} & +  0.41 x_{26} & +  0.23 x_{28} & -32.25 x_{7} & -1.20 x_{32} & +  1.97 x_{20} & +  2.78 x_{27} & -12.17 x_{8} & -12.14 x_{9} & -0.74 x_{10} & -24.64 x_{11} & -21.58 x_{12} & -43.01 x_{13} & + 20.81 x_{14} & -0.56 x_{25} & -15.41 x_{16} & + 24.84 x_{17}\\
 x_{21}   &  3.85925293917 & -9.93 x_{5} & -0.03 x_{26} & +  0.12 x_{28} & + 10.50 x_{7} & +  0.35 x_{32} & -0.31 x_{20} & -0.95 x_{27} & +  4.76 x_{8} & + 14.65 x_{9} & -6.61 x_{10} & + 17.97 x_{11} & +  2.94 x_{12} & + 15.94 x_{13} & -7.45 x_{14} & +  0.43 x_{25} & -3.53 x_{16} & -19.99 x_{17}\\
 x_{22}   &  2.74995709801 & +  5.30 x_{5} & +  0.12 x_{26} & -0.27 x_{28} & -26.59 x_{7} & -0.97 x_{32} & +  1.34 x_{20} & +  1.92 x_{27} & -1.18 x_{8} & +  1.31 x_{9} & +  1.55 x_{10} & -12.48 x_{11} & -30.15 x_{12} & -23.87 x_{13} & + 21.96 x_{14} & -0.38 x_{25} & -18.03 x_{16} & + 28.95 x_{17}\\
 x_{23}   &  1.67010145937 & -3.92 x_{5} & +  0.01 x_{26} & +  0.44 x_{28} & +  2.91 x_{7} & -0.96 x_{32} & +  0.30 x_{20} & +  0.57 x_{27} & -1.93 x_{8} & +  9.36 x_{9} & +  3.64 x_{10} & -15.90 x_{11} & -2.64 x_{12} & -25.34 x_{13} & + 16.88 x_{14} & -0.06 x_{25} & -11.44 x_{16} & -4.96 x_{17}\\
 x_{24}   &  2.65712640485 & -3.90 x_{5} & -0.12 x_{26} & +  0.53 x_{28} & + 20.22 x_{7} & +  0.61 x_{32} & -1.03 x_{20} & -2.33 x_{27} & -2.73 x_{8} & + 21.72 x_{9} & +  4.14 x_{10} & + 16.70 x_{11} & + 26.66 x_{12} & + 28.42 x_{13} & -24.40 x_{14} & +  0.49 x_{25} & + 16.75 x_{16} & -30.79 x_{17}\\
 x_{15}   &  0.180700555509 & -0.21 x_{5} & +  0.02 x_{26} & +  0.01 x_{28} & +  0.86 x_{7} & +  0.02 x_{32} & -0.03 x_{20} & -0.10 x_{27} & +  0.34 x_{8} & +  1.19 x_{9} & +  0.17 x_{10} & +  0.91 x_{11} & +  1.35 x_{12} & +  1.20 x_{13} & -0.49 x_{14} & +  0.06 x_{25} & +  0.70 x_{16} & -1.21 x_{17}\\
 x_{3}   &  0.0193838001202 & -0.32 x_{5} & +  0.02 x_{26} & +  0.01 x_{28} & +  1.63 x_{7} & +  0.08 x_{32} & -0.10 x_{20} & -0.13 x_{27} & +  0.08 x_{8} & +  0.61 x_{9} & +  0.02 x_{10} & +  1.50 x_{11} & +  1.28 x_{12} & +  2.34 x_{13} & -1.03 x_{14} & +  0.06 x_{25} & +  1.46 x_{16} & -1.69 x_{17}\\
 x_{6}   &  0.134185579752 & -0.39 x_{5} & -0.01 x_{26} & -0.04 x_{28} & -0.31 x_{7} & +  0.01 x_{32} & +  0.04 x_{20} & +  0.01 x_{27} & +  0.16 x_{8} & -0.21 x_{9} & +  0.13 x_{10} & +  0.72 x_{11} & -0.30 x_{12} & -0.10 x_{13} & +  0.36 x_{14} & +  0.03 x_{25} & +  0.05 x_{16} & +  0.30 x_{17}\\
 x_{2}   &  0.037591800387 & +  0.40 x_{5} & -0.02 x_{26} & +  0.05 x_{28} & +  0.30 x_{7} & -0.01 x_{32} & -0.03 x_{20} & +  0.03 x_{27} & -0.03 x_{8} & -0.04 x_{9} & +  0.76 x_{10} & -0.62 x_{11} & +  0.50 x_{12} & -0.39 x_{13} & +  0.03 x_{14} & +  0.03 x_{25} & +  0.34 x_{16} & -0.65 x_{17}\\
 x_{29}   &  1.76820415994 & -3.69 x_{5} & +  0.20 x_{26} & -0.68 x_{28} & -21.17 x_{7} & -1.07 x_{32} & +  1.43 x_{20} & +  1.41 x_{27} & +  9.13 x_{8} & -4.43 x_{9} & -3.38 x_{10} & -11.38 x_{11} & -22.80 x_{12} & -18.43 x_{13} & + 18.20 x_{14} & -0.22 x_{25} & -18.54 x_{16} & + 24.75 x_{17}\\
 x_{30}   &  4.60170098168 & +  4.00 x_{5} & +  0.03 x_{26} & +  0.25 x_{28} & -13.09 x_{7} & -0.91 x_{32} & +  0.39 x_{20} & +  0.67 x_{27} & +  6.79 x_{8} & +  1.89 x_{9} & + 13.93 x_{10} & -19.91 x_{11} & -1.05 x_{12} & -22.48 x_{13} & +  2.41 x_{14} & +  0.25 x_{25} & -12.57 x_{16} & +  5.10 x_{17}\\
 x_{31}   &  5.63584965035 & -9.53 x_{5} & +  0.17 x_{26} & -0.05 x_{28} & -3.45 x_{7} & -0.04 x_{32} & +  0.55 x_{20} & -0.18 x_{27} & +  5.05 x_{8} & +  7.06 x_{9} & +  5.21 x_{10} & +  1.37 x_{11} & + 10.24 x_{12} & +  1.56 x_{13} & +  5.75 x_{14} & +  0.20 x_{25} & -12.11 x_{16} & -3.41 x_{17}\\
 x_{4}   &  0.185080728162 & +  0.10 x_{5} & +  0.02 x_{26} & +  0.02 x_{28} & -1.70 x_{7} & -0.00 x_{32} & +  0.08 x_{20} & +  0.10 x_{27} & -0.83 x_{8} & -0.57 x_{9} & -0.79 x_{10} & -0.92 x_{11} & -1.29 x_{12} & -1.67 x_{13} & +  0.26 x_{14} & -0.05 x_{25} & -0.87 x_{16} & +  1.31 x_{17}\\
\hline
z    &  1.36426621794 & -1.99 x_{5} & +  0.24 x_{26} & +  0.11 x_{28} & -12.10 x_{7} & -0.23 x_{32} & +  0.36 x_{20} & +  0.54 x_{27} & +  1.62 x_{8} & -0.43 x_{9} & -1.42 x_{10} & -3.01 x_{11} & -4.80 x_{12} & -8.41 x_{13} & +  9.60 x_{14} & -0.14 x_{25} & -1.03 x_{16} & +  3.88 x_{17}\\
\end{array}\]


 $ x_{8} $ enters and $ x_{4} $ leaves 

 \[\begin{array}{c| c c@{\hskip 2pt} c@{\hskip 2pt} c@{\hskip 2pt} c@{\hskip 2pt} c@{\hskip 2pt} c@{\hskip 2pt} c@{\hskip 2pt} c@{\hskip 2pt} c@{\hskip 2pt} c@{\hskip 2pt} c@{\hskip 2pt} c@{\hskip 2pt} c@{\hskip 2pt} c@{\hskip 2pt} c@{\hskip 2pt} c@{\hskip 2pt} c@{\hskip 2pt} }
 x_{18}   &  3.34635875172 & -4.57 x_{5} & +  0.26 x_{26} & +  0.13 x_{28} & + 11.56 x_{7} & +  0.11 x_{32} & -0.95 x_{20} & -0.54 x_{27} & +  3.34 x_{4} & +  4.34 x_{9} & -1.23 x_{10} & +  9.05 x_{11} & +  2.55 x_{12} & +  8.42 x_{13} & +  2.43 x_{14} & +  0.68 x_{25} & +  9.17 x_{16} & -22.05 x_{17}\\
 x_{1}   &  0.0496799499407 & -0.66 x_{5} & +  0.03 x_{26} & -0.05 x_{28} & -1.43 x_{7} & -0.08 x_{32} & +  0.06 x_{20} & +  0.12 x_{27} & +  0.63 x_{4} & -0.20 x_{9} & +  0.15 x_{10} & -1.06 x_{11} & -1.55 x_{12} & -1.55 x_{13} & +  1.83 x_{14} & -0.01 x_{25} & -0.37 x_{16} & +  1.46 x_{17}\\
 x_{19}   &  1.01347913833 & +  0.37 x_{5} & +  0.10 x_{26} & -0.08 x_{28} & -7.25 x_{7} & -1.14 x_{32} & +  0.79 x_{20} & +  1.35 x_{27} & + 14.69 x_{4} & -3.80 x_{9} & + 10.92 x_{10} & -11.09 x_{11} & -2.65 x_{12} & -18.44 x_{13} & + 16.98 x_{14} & +  0.21 x_{25} & -2.56 x_{16} & +  5.63 x_{17}\\
 x_{21}   &  4.92325474565 & -9.38 x_{5} & +  0.09 x_{26} & +  0.23 x_{28} & +  0.71 x_{7} & +  0.32 x_{32} & +  0.16 x_{20} & -0.39 x_{27} & -5.75 x_{4} & + 11.38 x_{9} & -11.18 x_{10} & + 12.67 x_{11} & -4.47 x_{12} & +  6.32 x_{13} & -5.95 x_{14} & +  0.13 x_{25} & -8.56 x_{16} & -12.48 x_{17}\\
 x_{22}   &  2.48671398847 & +  5.16 x_{5} & +  0.10 x_{26} & -0.30 x_{28} & -24.17 x_{7} & -0.96 x_{32} & +  1.23 x_{20} & +  1.78 x_{27} & +  1.42 x_{4} & +  2.12 x_{9} & +  2.68 x_{10} & -11.17 x_{11} & -28.32 x_{12} & -21.49 x_{13} & + 21.59 x_{14} & -0.31 x_{25} & -16.79 x_{16} & + 27.09 x_{17}\\
 x_{23}   &  1.23946558184 & -4.14 x_{5} & -0.04 x_{26} & +  0.40 x_{28} & +  6.87 x_{7} & -0.95 x_{32} & +  0.11 x_{20} & +  0.34 x_{27} & +  2.33 x_{4} & + 10.68 x_{9} & +  5.49 x_{10} & -13.75 x_{11} & +  0.36 x_{12} & -21.45 x_{13} & + 16.27 x_{14} & +  0.06 x_{25} & -9.41 x_{16} & -8.00 x_{17}\\
 x_{24}   &  2.0475807225 & -4.21 x_{5} & -0.19 x_{26} & +  0.46 x_{28} & + 25.83 x_{7} & +  0.62 x_{32} & -1.30 x_{20} & -2.65 x_{27} & +  3.29 x_{4} & + 23.59 x_{9} & +  6.76 x_{10} & + 19.74 x_{11} & + 30.90 x_{12} & + 33.93 x_{13} & -25.26 x_{14} & +  0.66 x_{25} & + 19.63 x_{16} & -35.09 x_{17}\\
 x_{15}   &  0.25680831595 & -0.17 x_{5} & +  0.03 x_{26} & +  0.02 x_{28} & +  0.16 x_{7} & +  0.02 x_{32} & +  0.00 x_{20} & -0.06 x_{27} & -0.41 x_{4} & +  0.96 x_{9} & -0.15 x_{10} & +  0.53 x_{11} & +  0.82 x_{12} & +  0.51 x_{13} & -0.38 x_{14} & +  0.04 x_{25} & +  0.34 x_{16} & -0.67 x_{17}\\
 x_{3}   &  0.0377829148743 & -0.31 x_{5} & +  0.02 x_{26} & +  0.01 x_{28} & +  1.46 x_{7} & +  0.08 x_{32} & -0.09 x_{20} & -0.12 x_{27} & -0.10 x_{4} & +  0.55 x_{9} & -0.06 x_{10} & +  1.41 x_{11} & +  1.15 x_{12} & +  2.18 x_{13} & -1.01 x_{14} & +  0.05 x_{25} & +  1.38 x_{16} & -1.56 x_{17}\\
 x_{6}   &  0.170694424649 & -0.37 x_{5} & -0.01 x_{26} & -0.04 x_{28} & -0.65 x_{7} & +  0.01 x_{32} & +  0.06 x_{20} & +  0.03 x_{27} & -0.20 x_{4} & -0.32 x_{9} & -0.03 x_{10} & +  0.53 x_{11} & -0.56 x_{12} & -0.43 x_{13} & +  0.41 x_{14} & +  0.02 x_{25} & -0.13 x_{16} & +  0.56 x_{17}\\
 x_{2}   &  0.0318910902578 & +  0.40 x_{5} & -0.02 x_{26} & +  0.05 x_{28} & +  0.35 x_{7} & -0.01 x_{32} & -0.03 x_{20} & +  0.02 x_{27} & +  0.03 x_{4} & -0.02 x_{9} & +  0.78 x_{10} & -0.59 x_{11} & +  0.53 x_{12} & -0.34 x_{13} & +  0.02 x_{14} & +  0.03 x_{25} & +  0.36 x_{16} & -0.69 x_{17}\\
 x_{29}   &  3.80800175483 & -2.63 x_{5} & +  0.43 x_{26} & -0.45 x_{28} & -39.93 x_{7} & -1.12 x_{32} & +  2.32 x_{20} & +  2.48 x_{27} & -11.02 x_{4} & -10.69 x_{9} & -12.13 x_{10} & -21.54 x_{11} & -37.01 x_{12} & -36.86 x_{13} & + 21.08 x_{14} & -0.80 x_{25} & -28.18 x_{16} & + 39.16 x_{17}\\
 x_{30}   &  6.11910228091 & +  4.78 x_{5} & +  0.20 x_{26} & +  0.42 x_{28} & -27.04 x_{7} & -0.94 x_{32} & +  1.06 x_{20} & +  1.48 x_{27} & -8.20 x_{4} & -2.76 x_{9} & +  7.42 x_{10} & -27.48 x_{11} & -11.61 x_{12} & -36.19 x_{13} & +  4.55 x_{14} & -0.18 x_{25} & -19.74 x_{16} & + 15.82 x_{17}\\
 x_{31}   &  6.76338760841 & -8.95 x_{5} & +  0.30 x_{26} & +  0.08 x_{28} & -13.82 x_{7} & -0.07 x_{32} & +  1.04 x_{20} & +  0.41 x_{27} & -6.09 x_{4} & +  3.60 x_{9} & +  0.37 x_{10} & -4.25 x_{11} & +  2.39 x_{12} & -8.63 x_{13} & +  7.34 x_{14} & -0.12 x_{25} & -17.44 x_{16} & +  4.56 x_{17}\\
 x_{8}   &  0.22343208321 & +  0.12 x_{5} & +  0.03 x_{26} & +  0.03 x_{28} & -2.05 x_{7} & -0.01 x_{32} & +  0.10 x_{20} & +  0.12 x_{27} & -1.21 x_{4} & -0.69 x_{9} & -0.96 x_{10} & -1.11 x_{11} & -1.56 x_{12} & -2.02 x_{13} & +  0.31 x_{14} & -0.06 x_{25} & -1.06 x_{16} & +  1.58 x_{17}\\
\hline
z    &  1.72712745468 & -1.81 x_{5} & +  0.28 x_{26} & +  0.15 x_{28} & -15.43 x_{7} & -0.24 x_{32} & +  0.52 x_{20} & +  0.73 x_{27} & -1.96 x_{4} & -1.54 x_{9} & -2.98 x_{10} & -4.82 x_{11} & -7.33 x_{12} & -11.69 x_{13} & + 10.11 x_{14} & -0.24 x_{25} & -2.74 x_{16} & +  6.45 x_{17}\\
\end{array}\]


 $ x_{14} $ enters and $ x_{3} $ leaves 

 \[\begin{array}{c| c c@{\hskip 2pt} c@{\hskip 2pt} c@{\hskip 2pt} c@{\hskip 2pt} c@{\hskip 2pt} c@{\hskip 2pt} c@{\hskip 2pt} c@{\hskip 2pt} c@{\hskip 2pt} c@{\hskip 2pt} c@{\hskip 2pt} c@{\hskip 2pt} c@{\hskip 2pt} c@{\hskip 2pt} c@{\hskip 2pt} c@{\hskip 2pt} c@{\hskip 2pt} }
 x_{18}   &  3.43759160055 & -5.33 x_{5} & +  0.32 x_{26} & +  0.15 x_{28} & + 15.09 x_{7} & +  0.30 x_{32} & -1.18 x_{20} & -0.83 x_{27} & +  3.10 x_{4} & +  5.67 x_{9} & -1.38 x_{10} & + 12.46 x_{11} & +  5.33 x_{12} & + 13.67 x_{13} & -2.41 x_{3} & +  0.80 x_{25} & + 12.49 x_{16} & -25.81 x_{17}\\
 x_{1}   &  0.118434767777 & -1.23 x_{5} & +  0.07 x_{26} & -0.04 x_{28} & +  1.23 x_{7} & +  0.06 x_{32} & -0.11 x_{20} & -0.10 x_{27} & +  0.45 x_{4} & +  0.80 x_{9} & +  0.03 x_{10} & +  1.50 x_{11} & +  0.55 x_{12} & +  2.41 x_{13} & -1.82 x_{3} & +  0.09 x_{25} & +  2.14 x_{16} & -1.38 x_{17}\\
 x_{19}   &  1.65103508617 & -4.90 x_{5} & +  0.50 x_{26} & +  0.05 x_{28} & + 17.39 x_{7} & +  0.15 x_{32} & -0.79 x_{20} & -0.68 x_{27} & + 13.01 x_{4} & +  5.52 x_{9} & +  9.88 x_{10} & + 12.70 x_{11} & + 16.82 x_{12} & + 18.28 x_{13} & -16.87 x_{3} & +  1.08 x_{25} & + 20.68 x_{16} & -20.66 x_{17}\\
 x_{21}   &  4.69972551423 & -7.53 x_{5} & -0.05 x_{26} & +  0.19 x_{28} & -7.93 x_{7} & -0.13 x_{32} & +  0.71 x_{20} & +  0.32 x_{27} & -5.16 x_{4} & +  8.12 x_{9} & -10.81 x_{10} & +  4.33 x_{11} & -11.30 x_{12} & -6.55 x_{13} & +  5.92 x_{3} & -0.17 x_{25} & -16.70 x_{16} & -3.26 x_{17}\\
 x_{22}   &  3.29726114368 & -1.53 x_{5} & +  0.60 x_{26} & -0.13 x_{28} & +  7.17 x_{7} & +  0.67 x_{32} & -0.77 x_{20} & -0.80 x_{27} & -0.71 x_{4} & + 13.97 x_{9} & +  1.36 x_{10} & + 19.07 x_{11} & -3.56 x_{12} & + 25.19 x_{13} & -21.45 x_{3} & +  0.80 x_{25} & + 12.76 x_{16} & -6.33 x_{17}\\
 x_{23}   &  1.85028522515 & -9.19 x_{5} & +  0.35 x_{26} & +  0.52 x_{28} & + 30.49 x_{7} & +  0.28 x_{32} & -1.40 x_{20} & -1.60 x_{27} & +  0.72 x_{4} & + 19.60 x_{9} & +  4.49 x_{10} & +  9.03 x_{11} & + 19.02 x_{12} & + 13.73 x_{13} & -16.17 x_{3} & +  0.89 x_{25} & + 12.86 x_{16} & -33.19 x_{17}\\
 x_{24}   &  1.09938185425 & +  3.62 x_{5} & -0.78 x_{26} & +  0.26 x_{28} & -10.83 x_{7} & -1.28 x_{32} & +  1.04 x_{20} & +  0.37 x_{27} & +  5.79 x_{4} & +  9.73 x_{9} & +  8.30 x_{10} & -15.63 x_{11} & +  1.94 x_{12} & -20.68 x_{13} & + 25.10 x_{3} & -0.64 x_{25} & -14.93 x_{16} & +  4.01 x_{17}\\
 x_{15}   &  0.242425582453 & -0.05 x_{5} & +  0.02 x_{26} & +  0.02 x_{28} & -0.39 x_{7} & -0.01 x_{32} & +  0.04 x_{20} & -0.02 x_{27} & -0.37 x_{4} & +  0.75 x_{9} & -0.13 x_{10} & -0.01 x_{11} & +  0.38 x_{12} & -0.32 x_{13} & +  0.38 x_{3} & +  0.02 x_{25} & -0.18 x_{16} & -0.08 x_{17}\\
 x_{14}   &  0.0375404305857 & -0.31 x_{5} & +  0.02 x_{26} & +  0.01 x_{28} & +  1.45 x_{7} & +  0.08 x_{32} & -0.09 x_{20} & -0.12 x_{27} & -0.10 x_{4} & +  0.55 x_{9} & -0.06 x_{10} & +  1.40 x_{11} & +  1.15 x_{12} & +  2.16 x_{13} & -0.99 x_{3} & +  0.05 x_{25} & +  1.37 x_{16} & -1.55 x_{17}\\
 x_{6}   &  0.186198640522 & -0.50 x_{5} & +  0.00 x_{26} & -0.03 x_{28} & -0.05 x_{7} & +  0.04 x_{32} & +  0.02 x_{20} & -0.02 x_{27} & -0.24 x_{4} & -0.10 x_{9} & -0.05 x_{10} & +  1.11 x_{11} & -0.08 x_{12} & +  0.46 x_{13} & -0.41 x_{3} & +  0.04 x_{25} & +  0.44 x_{16} & -0.08 x_{17}\\
 x_{2}   &  0.0326919189367 & +  0.39 x_{5} & -0.02 x_{26} & +  0.05 x_{28} & +  0.38 x_{7} & -0.00 x_{32} & -0.03 x_{20} & +  0.02 x_{27} & +  0.03 x_{4} & -0.01 x_{9} & +  0.78 x_{10} & -0.56 x_{11} & +  0.56 x_{12} & -0.29 x_{13} & -0.02 x_{3} & +  0.03 x_{25} & +  0.39 x_{16} & -0.73 x_{17}\\
 x_{29}   &  4.59925708799 & -9.17 x_{5} & +  0.92 x_{26} & -0.29 x_{28} & -9.34 x_{7} & +  0.48 x_{32} & +  0.36 x_{20} & -0.03 x_{27} & -13.10 x_{4} & +  0.88 x_{9} & -13.43 x_{10} & +  7.97 x_{11} & -12.83 x_{12} & +  8.71 x_{13} & -20.94 x_{3} & +  0.28 x_{25} & +  0.66 x_{16} & +  6.53 x_{17}\\
 x_{30}   &  6.28982254763 & +  3.38 x_{5} & +  0.30 x_{26} & +  0.46 x_{28} & -20.44 x_{7} & -0.60 x_{32} & +  0.63 x_{20} & +  0.93 x_{27} & -8.65 x_{4} & -0.27 x_{9} & +  7.14 x_{10} & -21.11 x_{11} & -6.40 x_{12} & -26.36 x_{13} & -4.52 x_{3} & +  0.05 x_{25} & -13.52 x_{16} & +  8.78 x_{17}\\
 x_{31}   &  7.03900603932 & -11.22 x_{5} & +  0.47 x_{26} & +  0.14 x_{28} & -3.16 x_{7} & +  0.49 x_{32} & +  0.36 x_{20} & -0.46 x_{27} & -6.82 x_{4} & +  7.63 x_{9} & -0.08 x_{10} & +  6.03 x_{11} & + 10.81 x_{12} & +  7.25 x_{13} & -7.29 x_{3} & +  0.26 x_{25} & -7.40 x_{16} & -6.81 x_{17}\\
 x_{8}   &  0.235252312124 & +  0.02 x_{5} & +  0.03 x_{26} & +  0.03 x_{28} & -1.60 x_{7} & +  0.02 x_{32} & +  0.07 x_{20} & +  0.08 x_{27} & -1.24 x_{4} & -0.51 x_{9} & -0.98 x_{10} & -0.67 x_{11} & -1.19 x_{12} & -1.34 x_{13} & -0.31 x_{3} & -0.05 x_{25} & -0.63 x_{16} & +  1.09 x_{17}\\
\hline
z    &  2.10683940213 & -4.94 x_{5} & +  0.51 x_{26} & +  0.23 x_{28} & -0.76 x_{7} & +  0.52 x_{32} & -0.42 x_{20} & -0.48 x_{27} & -2.96 x_{4} & +  4.01 x_{9} & -3.60 x_{10} & +  9.34 x_{11} & +  4.27 x_{12} & + 10.18 x_{13} & -10.05 x_{3} & +  0.28 x_{25} & + 11.10 x_{16} & -9.21 x_{17}\\
\end{array}\]


 $ x_{9} $ enters and $ x_{8} $ leaves 

 \[\begin{array}{c| c c@{\hskip 2pt} c@{\hskip 2pt} c@{\hskip 2pt} c@{\hskip 2pt} c@{\hskip 2pt} c@{\hskip 2pt} c@{\hskip 2pt} c@{\hskip 2pt} c@{\hskip 2pt} c@{\hskip 2pt} c@{\hskip 2pt} c@{\hskip 2pt} c@{\hskip 2pt} c@{\hskip 2pt} c@{\hskip 2pt} c@{\hskip 2pt} c@{\hskip 2pt} }
 x_{18}   &  6.0418518724 & -5.13 x_{5} & +  0.68 x_{26} & +  0.46 x_{28} & -2.60 x_{7} & +  0.50 x_{32} & -0.42 x_{20} & +  0.06 x_{27} & -10.61 x_{4} & -11.07 x_{8} & -12.20 x_{10} & +  5.01 x_{11} & -7.89 x_{12} & -1.14 x_{13} & -5.88 x_{3} & +  0.29 x_{25} & +  5.57 x_{16} & -13.73 x_{17}\\
 x_{1}   &  0.486012728592 & -1.20 x_{5} & +  0.12 x_{26} & +  0.01 x_{28} & -1.27 x_{7} & +  0.09 x_{32} & -0.00 x_{20} & +  0.03 x_{27} & -1.49 x_{4} & -1.56 x_{8} & -1.49 x_{10} & +  0.45 x_{11} & -1.32 x_{12} & +  0.31 x_{13} & -2.31 x_{3} & +  0.01 x_{25} & +  1.16 x_{16} & +  0.33 x_{17}\\
 x_{19}   &  4.18482350741 & -4.71 x_{5} & +  0.85 x_{26} & +  0.35 x_{28} & +  0.19 x_{7} & +  0.35 x_{32} & -0.05 x_{20} & +  0.19 x_{27} & -0.33 x_{4} & -10.77 x_{8} & -0.65 x_{10} & +  5.45 x_{11} & +  3.96 x_{12} & +  3.86 x_{13} & -20.24 x_{3} & +  0.58 x_{25} & + 13.95 x_{16} & -8.91 x_{17}\\
 x_{21}   &  8.42688252254 & -7.25 x_{5} & +  0.46 x_{26} & +  0.63 x_{28} & -33.24 x_{7} & +  0.17 x_{32} & +  1.79 x_{20} & +  1.60 x_{27} & -24.78 x_{4} & -15.84 x_{8} & -26.30 x_{10} & -6.33 x_{11} & -30.23 x_{12} & -27.75 x_{13} & +  0.96 x_{3} & -0.91 x_{25} & -26.61 x_{16} & + 14.03 x_{17}\\
 x_{22}   &  9.70969193685 & -1.04 x_{5} & +  1.49 x_{26} & +  0.62 x_{28} & -36.38 x_{7} & +  1.18 x_{32} & +  1.09 x_{20} & +  1.40 x_{27} & -34.46 x_{4} & -27.26 x_{8} & -25.30 x_{10} & +  0.73 x_{11} & -36.12 x_{12} & -11.28 x_{13} & -29.98 x_{3} & -0.48 x_{25} & -4.29 x_{16} & + 23.41 x_{17}\\
 x_{23}   &  10.851179871 & -8.51 x_{5} & +  1.59 x_{26} & +  1.58 x_{28} & -30.64 x_{7} & +  1.00 x_{32} & +  1.21 x_{20} & +  1.48 x_{27} & -46.66 x_{4} & -38.26 x_{8} & -32.92 x_{10} & -16.71 x_{11} & -26.69 x_{12} & -37.46 x_{13} & -28.14 x_{3} & -0.90 x_{25} & -11.07 x_{16} & +  8.56 x_{17}\\
 x_{24}   &  5.56925821122 & +  3.96 x_{5} & -0.17 x_{26} & +  0.79 x_{28} & -41.18 x_{7} & -0.93 x_{32} & +  2.34 x_{20} & +  1.90 x_{27} & -17.74 x_{4} & -19.00 x_{8} & -10.28 x_{10} & -28.42 x_{11} & -20.76 x_{12} & -46.10 x_{13} & + 19.15 x_{3} & -1.53 x_{25} & -26.81 x_{16} & + 24.75 x_{17}\\
 x_{15}   &  0.585570700812 & -0.03 x_{5} & +  0.06 x_{26} & +  0.06 x_{28} & -2.72 x_{7} & +  0.01 x_{32} & +  0.14 x_{20} & +  0.10 x_{27} & -2.18 x_{4} & -1.46 x_{8} & -1.56 x_{10} & -0.99 x_{11} & -1.36 x_{12} & -2.27 x_{13} & -0.08 x_{3} & -0.05 x_{25} & -1.09 x_{16} & +  1.51 x_{17}\\
 x_{14}   &  0.289467573327 & -0.29 x_{5} & +  0.06 x_{26} & +  0.04 x_{28} & -0.26 x_{7} & +  0.10 x_{32} & -0.02 x_{20} & -0.03 x_{27} & -1.42 x_{4} & -1.07 x_{8} & -1.11 x_{10} & +  0.68 x_{11} & -0.13 x_{12} & +  0.73 x_{13} & -1.33 x_{3} & +  0.00 x_{25} & +  0.70 x_{16} & -0.38 x_{17}\\
 x_{6}   &  0.14189276665 & -0.50 x_{5} & -0.00 x_{26} & -0.04 x_{28} & +  0.25 x_{7} & +  0.04 x_{32} & +  0.01 x_{20} & -0.04 x_{27} & -0.00 x_{4} & +  0.19 x_{8} & +  0.13 x_{10} & +  1.24 x_{11} & +  0.14 x_{12} & +  0.71 x_{13} & -0.35 x_{3} & +  0.05 x_{25} & +  0.56 x_{16} & -0.29 x_{17}\\
 x_{2}   &  0.0298096465216 & +  0.39 x_{5} & -0.02 x_{26} & +  0.05 x_{28} & +  0.40 x_{7} & -0.00 x_{32} & -0.03 x_{20} & +  0.02 x_{27} & +  0.04 x_{4} & +  0.01 x_{8} & +  0.79 x_{10} & -0.55 x_{11} & +  0.57 x_{12} & -0.28 x_{13} & -0.02 x_{3} & +  0.03 x_{25} & +  0.40 x_{16} & -0.74 x_{17}\\
 x_{29}   &  5.00292219201 & -9.14 x_{5} & +  0.98 x_{26} & -0.24 x_{28} & -12.08 x_{7} & +  0.51 x_{32} & +  0.48 x_{20} & +  0.11 x_{27} & -15.23 x_{4} & -1.72 x_{8} & -15.11 x_{10} & +  6.82 x_{11} & -14.88 x_{12} & +  6.41 x_{13} & -21.48 x_{3} & +  0.20 x_{25} & -0.41 x_{16} & +  8.40 x_{17}\\
 x_{30}   &  6.16775067249 & +  3.37 x_{5} & +  0.29 x_{26} & +  0.44 x_{28} & -19.62 x_{7} & -0.61 x_{32} & +  0.60 x_{20} & +  0.89 x_{27} & -8.01 x_{4} & +  0.52 x_{8} & +  7.65 x_{10} & -20.76 x_{11} & -5.78 x_{12} & -25.67 x_{13} & -4.36 x_{3} & +  0.08 x_{25} & -13.20 x_{16} & +  8.21 x_{17}\\
 x_{31}   &  10.5426933897 & -10.96 x_{5} & +  0.96 x_{26} & +  0.55 x_{28} & -26.95 x_{7} & +  0.77 x_{32} & +  1.37 x_{20} & +  0.74 x_{27} & -25.26 x_{4} & -14.89 x_{8} & -14.65 x_{10} & -3.99 x_{11} & -6.98 x_{12} & -12.68 x_{13} & -11.95 x_{3} & -0.44 x_{25} & -16.71 x_{16} & +  9.44 x_{17}\\
 x_{9}   &  0.459160208869 & +  0.03 x_{5} & +  0.06 x_{26} & +  0.05 x_{28} & -3.12 x_{7} & +  0.04 x_{32} & +  0.13 x_{20} & +  0.16 x_{27} & -2.42 x_{4} & -1.95 x_{8} & -1.91 x_{10} & -1.31 x_{11} & -2.33 x_{12} & -2.61 x_{13} & -0.61 x_{3} & -0.09 x_{25} & -1.22 x_{16} & +  2.13 x_{17}\\
\hline
z    &  3.9489520592 & -4.80 x_{5} & +  0.77 x_{26} & +  0.45 x_{28} & -13.27 x_{7} & +  0.67 x_{32} & +  0.12 x_{20} & +  0.15 x_{27} & -12.66 x_{4} & -7.83 x_{8} & -11.25 x_{10} & +  4.07 x_{11} & -5.08 x_{12} & -0.30 x_{13} & -12.50 x_{3} & -0.09 x_{25} & +  6.20 x_{16} & -0.67 x_{17}\\
\end{array}\]


 $ x_{11} $ enters and $ x_{2} $ leaves 

 \[\begin{array}{c| c c@{\hskip 2pt} c@{\hskip 2pt} c@{\hskip 2pt} c@{\hskip 2pt} c@{\hskip 2pt} c@{\hskip 2pt} c@{\hskip 2pt} c@{\hskip 2pt} c@{\hskip 2pt} c@{\hskip 2pt} c@{\hskip 2pt} c@{\hskip 2pt} c@{\hskip 2pt} c@{\hskip 2pt} c@{\hskip 2pt} c@{\hskip 2pt} c@{\hskip 2pt} }
 x_{18}   &  6.31265282996 & -1.58 x_{5} & +  0.52 x_{26} & +  0.89 x_{28} & +  1.02 x_{7} & +  0.46 x_{32} & -0.74 x_{20} & +  0.24 x_{27} & -10.21 x_{4} & -10.96 x_{8} & -5.00 x_{10} & -9.08 x_{2} & -2.68 x_{12} & -3.64 x_{13} & -6.04 x_{3} & +  0.57 x_{25} & +  9.20 x_{16} & -20.45 x_{17}\\
 x_{1}   &  0.510452078488 & -0.88 x_{5} & +  0.11 x_{26} & +  0.04 x_{28} & -0.94 x_{7} & +  0.09 x_{32} & -0.03 x_{20} & +  0.04 x_{27} & -1.45 x_{4} & -1.55 x_{8} & -0.84 x_{10} & -0.82 x_{2} & -0.85 x_{12} & +  0.09 x_{13} & -2.32 x_{3} & +  0.04 x_{25} & +  1.49 x_{16} & -0.28 x_{17}\\
 x_{19}   &  4.47952078488 & -0.84 x_{5} & +  0.68 x_{26} & +  0.82 x_{28} & +  4.13 x_{7} & +  0.30 x_{32} & -0.39 x_{20} & +  0.39 x_{27} & +  0.11 x_{4} & -10.65 x_{8} & +  7.19 x_{10} & -9.89 x_{2} & +  9.63 x_{12} & +  1.14 x_{13} & -20.42 x_{3} & +  0.89 x_{25} & + 17.90 x_{16} & -16.22 x_{17}\\
 x_{21}   &  8.08477962665 & -11.73 x_{5} & +  0.66 x_{26} & +  0.08 x_{28} & -37.82 x_{7} & +  0.22 x_{32} & +  2.18 x_{20} & +  1.37 x_{27} & -25.28 x_{4} & -15.98 x_{8} & -35.41 x_{10} & + 11.48 x_{2} & -36.81 x_{12} & -24.59 x_{13} & +  1.16 x_{3} & -1.28 x_{25} & -31.20 x_{16} & + 22.52 x_{17}\\
 x_{22}   &  9.7491501163 & -0.53 x_{5} & +  1.46 x_{26} & +  0.69 x_{28} & -35.85 x_{7} & +  1.17 x_{32} & +  1.04 x_{20} & +  1.43 x_{27} & -34.41 x_{4} & -27.24 x_{8} & -24.25 x_{10} & -1.32 x_{2} & -35.36 x_{12} & -11.64 x_{13} & -30.00 x_{3} & -0.44 x_{25} & -3.76 x_{16} & + 22.43 x_{17}\\
 x_{23}   &  9.9476799666 & -20.35 x_{5} & +  2.11 x_{26} & +  0.15 x_{28} & -42.72 x_{7} & +  1.14 x_{32} & +  2.25 x_{20} & +  0.87 x_{27} & -47.99 x_{4} & -38.63 x_{8} & -56.97 x_{10} & + 30.31 x_{2} & -44.09 x_{12} & -29.12 x_{13} & -27.61 x_{3} & -1.86 x_{25} & -23.19 x_{16} & + 30.99 x_{17}\\
 x_{24}   &  4.03275034294 & -16.19 x_{5} & +  0.72 x_{26} & -1.64 x_{28} & -61.73 x_{7} & -0.70 x_{32} & +  4.11 x_{20} & +  0.87 x_{27} & -20.00 x_{4} & -19.63 x_{8} & -51.16 x_{10} & + 51.54 x_{2} & -50.34 x_{12} & -31.91 x_{13} & + 20.05 x_{3} & -3.16 x_{25} & -47.42 x_{16} & + 62.89 x_{17}\\
 x_{15}   &  0.53210174748 & -0.73 x_{5} & +  0.09 x_{26} & -0.03 x_{28} & -3.44 x_{7} & +  0.02 x_{32} & +  0.20 x_{20} & +  0.06 x_{27} & -2.26 x_{4} & -1.48 x_{8} & -2.98 x_{10} & +  1.79 x_{2} & -2.39 x_{12} & -1.77 x_{13} & -0.04 x_{3} & -0.11 x_{25} & -1.81 x_{16} & +  2.84 x_{17}\\
 x_{14}   &  0.326228603805 & +  0.19 x_{5} & +  0.04 x_{26} & +  0.10 x_{28} & +  0.23 x_{7} & +  0.09 x_{32} & -0.06 x_{20} & -0.01 x_{27} & -1.37 x_{4} & -1.06 x_{8} & -0.13 x_{10} & -1.23 x_{2} & +  0.58 x_{12} & +  0.39 x_{13} & -1.35 x_{3} & +  0.04 x_{25} & +  1.19 x_{16} & -1.29 x_{17}\\
 x_{6}   &  0.208892467347 & +  0.38 x_{5} & -0.04 x_{26} & +  0.07 x_{28} & +  1.15 x_{7} & +  0.03 x_{32} & -0.07 x_{20} & +  0.01 x_{27} & +  0.09 x_{4} & +  0.22 x_{8} & +  1.91 x_{10} & -2.25 x_{2} & +  1.43 x_{12} & +  0.09 x_{13} & -0.39 x_{3} & +  0.12 x_{25} & +  1.46 x_{16} & -1.95 x_{17}\\
 x_{11}   &  0.0540719866404 & +  0.71 x_{5} & -0.03 x_{26} & +  0.09 x_{28} & +  0.72 x_{7} & -0.01 x_{32} & -0.06 x_{20} & +  0.04 x_{27} & +  0.08 x_{4} & +  0.02 x_{8} & +  1.44 x_{10} & -1.81 x_{2} & +  1.04 x_{12} & -0.50 x_{13} & -0.03 x_{3} & +  0.06 x_{25} & +  0.73 x_{16} & -1.34 x_{17}\\
 x_{29}   &  5.37158555496 & -4.30 x_{5} & +  0.76 x_{26} & +  0.34 x_{28} & -7.15 x_{7} & +  0.45 x_{32} & +  0.06 x_{20} & +  0.35 x_{27} & -14.69 x_{4} & -1.56 x_{8} & -5.30 x_{10} & -12.37 x_{2} & -7.79 x_{12} & +  3.01 x_{13} & -21.69 x_{3} & +  0.59 x_{25} & +  4.53 x_{16} & -0.75 x_{17}\\
 x_{30}   &  5.04531967555 & -11.35 x_{5} & +  0.94 x_{26} & -1.33 x_{28} & -34.63 x_{7} & -0.44 x_{32} & +  1.89 x_{20} & +  0.14 x_{27} & -9.66 x_{4} & +  0.06 x_{8} & -22.22 x_{10} & + 37.65 x_{2} & -27.38 x_{12} & -15.30 x_{13} & -3.70 x_{3} & -1.11 x_{25} & -28.25 x_{16} & + 36.07 x_{17}\\
 x_{31}   &  10.3267504622 & -13.79 x_{5} & +  1.08 x_{26} & +  0.21 x_{28} & -29.84 x_{7} & +  0.80 x_{32} & +  1.62 x_{20} & +  0.59 x_{27} & -25.58 x_{4} & -14.98 x_{8} & -20.39 x_{10} & +  7.24 x_{2} & -11.14 x_{12} & -10.69 x_{13} & -11.83 x_{3} & -0.67 x_{25} & -19.61 x_{16} & + 14.80 x_{17}\\
 x_{9}   &  0.388158287112 & -0.90 x_{5} & +  0.10 x_{26} & -0.06 x_{28} & -4.07 x_{7} & +  0.05 x_{32} & +  0.21 x_{20} & +  0.11 x_{27} & -2.52 x_{4} & -1.98 x_{8} & -3.80 x_{10} & +  2.38 x_{2} & -3.70 x_{12} & -1.96 x_{13} & -0.57 x_{3} & -0.17 x_{25} & -2.17 x_{16} & +  3.89 x_{17}\\
\hline
z    &  4.16914922169 & -1.91 x_{5} & +  0.64 x_{26} & +  0.80 x_{28} & -10.32 x_{7} & +  0.64 x_{32} & -0.14 x_{20} & +  0.30 x_{27} & -12.33 x_{4} & -7.74 x_{8} & -5.40 x_{10} & -7.39 x_{2} & -0.85 x_{12} & -2.33 x_{13} & -12.63 x_{3} & +  0.14 x_{25} & +  9.16 x_{16} & -6.14 x_{17}\\
\end{array}\]


 $ x_{16} $ enters and $ x_{24} $ leaves 

 \[\begin{array}{c| c c@{\hskip 2pt} c@{\hskip 2pt} c@{\hskip 2pt} c@{\hskip 2pt} c@{\hskip 2pt} c@{\hskip 2pt} c@{\hskip 2pt} c@{\hskip 2pt} c@{\hskip 2pt} c@{\hskip 2pt} c@{\hskip 2pt} c@{\hskip 2pt} c@{\hskip 2pt} c@{\hskip 2pt} c@{\hskip 2pt} c@{\hskip 2pt} c@{\hskip 2pt} }
 x_{18}   &  7.09529068312 & -4.72 x_{5} & +  0.66 x_{26} & +  0.57 x_{28} & -10.96 x_{7} & +  0.33 x_{32} & +  0.06 x_{20} & +  0.41 x_{27} & -14.09 x_{4} & -14.77 x_{8} & -14.93 x_{10} & +  0.92 x_{2} & -12.45 x_{12} & -9.83 x_{13} & -2.15 x_{3} & -0.04 x_{25} & -0.19 x_{24} & -8.24 x_{17}\\
 x_{1}   &  0.6371702843 & -1.39 x_{5} & +  0.13 x_{26} & -0.01 x_{28} & -2.88 x_{7} & +  0.06 x_{32} & +  0.10 x_{20} & +  0.07 x_{27} & -2.08 x_{4} & -2.17 x_{8} & -2.45 x_{10} & +  0.80 x_{2} & -2.43 x_{12} & -0.91 x_{13} & -1.69 x_{3} & -0.06 x_{25} & -0.03 x_{24} & +  1.70 x_{17}\\
 x_{19}   &  6.0018486468 & -6.95 x_{5} & +  0.95 x_{26} & +  0.20 x_{28} & -19.18 x_{7} & +  0.04 x_{32} & +  1.16 x_{20} & +  0.71 x_{27} & -7.44 x_{4} & -18.06 x_{8} & -12.13 x_{10} & +  9.57 x_{2} & -9.37 x_{12} & -10.90 x_{13} & -12.85 x_{3} & -0.30 x_{25} & -0.38 x_{24} & +  7.52 x_{17}\\
 x_{21}   &  5.43127036845 & -1.08 x_{5} & +  0.18 x_{26} & +  1.17 x_{28} & +  2.80 x_{7} & +  0.68 x_{32} & -0.52 x_{20} & +  0.80 x_{27} & -12.12 x_{4} & -3.07 x_{8} & -1.74 x_{10} & -22.44 x_{2} & -3.69 x_{12} & -3.59 x_{13} & -12.03 x_{3} & +  0.80 x_{25} & +  0.66 x_{24} & -18.86 x_{17}\\
 x_{22}   &  9.42946841735 & +  0.76 x_{5} & +  1.41 x_{26} & +  0.82 x_{28} & -30.96 x_{7} & +  1.23 x_{32} & +  0.71 x_{20} & +  1.36 x_{27} & -32.82 x_{4} & -25.69 x_{8} & -20.19 x_{10} & -5.41 x_{2} & -31.37 x_{12} & -9.11 x_{13} & -31.59 x_{3} & -0.19 x_{25} & +  0.08 x_{24} & + 17.45 x_{17}\\
 x_{23}   &  7.97563993555 & -12.44 x_{5} & +  1.76 x_{26} & +  0.96 x_{28} & -12.54 x_{7} & +  1.48 x_{32} & +  0.24 x_{20} & +  0.45 x_{27} & -38.21 x_{4} & -29.03 x_{8} & -31.95 x_{10} & +  5.10 x_{2} & -19.47 x_{12} & -13.52 x_{13} & -37.41 x_{3} & -0.31 x_{25} & +  0.49 x_{24} & +  0.24 x_{17}\\
 x_{16}   &  0.0850486012671 & -0.34 x_{5} & +  0.02 x_{26} & -0.03 x_{28} & -1.30 x_{7} & -0.01 x_{32} & +  0.09 x_{20} & +  0.02 x_{27} & -0.42 x_{4} & -0.41 x_{8} & -1.08 x_{10} & +  1.09 x_{2} & -1.06 x_{12} & -0.67 x_{13} & +  0.42 x_{3} & -0.07 x_{25} & -0.02 x_{24} & +  1.33 x_{17}\\
 x_{15}   &  0.378002832558 & -0.11 x_{5} & +  0.07 x_{26} & +  0.04 x_{28} & -1.08 x_{7} & +  0.05 x_{32} & +  0.04 x_{20} & +  0.03 x_{27} & -1.49 x_{4} & -0.73 x_{8} & -1.02 x_{10} & -0.18 x_{2} & -0.47 x_{12} & -0.55 x_{13} & -0.81 x_{3} & +  0.01 x_{25} & +  0.04 x_{24} & +  0.44 x_{17}\\
 x_{14}   &  0.427579835505 & -0.22 x_{5} & +  0.06 x_{26} & +  0.05 x_{28} & -1.32 x_{7} & +  0.07 x_{32} & +  0.04 x_{20} & +  0.01 x_{27} & -1.87 x_{4} & -1.55 x_{8} & -1.42 x_{10} & +  0.06 x_{2} & -0.69 x_{12} & -0.41 x_{13} & -0.85 x_{3} & -0.04 x_{25} & -0.03 x_{24} & +  0.29 x_{17}\\
 x_{6}   &  0.332668692558 & -0.12 x_{5} & -0.02 x_{26} & +  0.02 x_{28} & -0.75 x_{7} & +  0.01 x_{32} & +  0.06 x_{20} & +  0.04 x_{27} & -0.52 x_{4} & -0.39 x_{8} & +  0.34 x_{10} & -0.67 x_{2} & -0.12 x_{12} & -0.89 x_{13} & +  0.22 x_{3} & +  0.03 x_{25} & -0.03 x_{24} & -0.02 x_{17}\\
 x_{11}   &  0.115741986891 & +  0.46 x_{5} & -0.02 x_{26} & +  0.06 x_{28} & -0.22 x_{7} & -0.02 x_{32} & +  0.00 x_{20} & +  0.05 x_{27} & -0.23 x_{4} & -0.28 x_{8} & +  0.66 x_{10} & -1.03 x_{2} & +  0.27 x_{12} & -0.99 x_{13} & +  0.28 x_{3} & +  0.01 x_{25} & -0.02 x_{24} & -0.38 x_{17}\\
 x_{29}   &  5.75688863845 & -5.85 x_{5} & +  0.83 x_{26} & +  0.18 x_{28} & -13.05 x_{7} & +  0.39 x_{32} & +  0.45 x_{20} & +  0.44 x_{27} & -16.60 x_{4} & -3.44 x_{8} & -10.18 x_{10} & -7.44 x_{2} & -12.60 x_{12} & -0.04 x_{13} & -19.78 x_{3} & +  0.29 x_{25} & -0.10 x_{24} & +  5.26 x_{17}\\
 x_{30}   &  2.64293382354 & -1.71 x_{5} & +  0.51 x_{26} & -0.35 x_{28} & +  2.15 x_{7} & -0.02 x_{32} & -0.56 x_{20} & -0.38 x_{27} & +  2.26 x_{4} & + 11.75 x_{8} & +  8.26 x_{10} & +  6.95 x_{2} & +  2.60 x_{12} & +  3.71 x_{13} & -15.64 x_{3} & +  0.77 x_{25} & +  0.60 x_{24} & -1.39 x_{17}\\
 x_{31}   &  8.65932716049 & -7.09 x_{5} & +  0.78 x_{26} & +  0.89 x_{28} & -4.32 x_{7} & +  1.09 x_{32} & -0.08 x_{20} & +  0.23 x_{27} & -17.31 x_{4} & -6.86 x_{8} & +  0.76 x_{10} & -14.07 x_{2} & +  9.67 x_{12} & +  2.51 x_{13} & -20.12 x_{3} & +  0.64 x_{25} & +  0.41 x_{24} & -11.20 x_{17}\\
 x_{9}   &  0.203366398714 & -0.15 x_{5} & +  0.07 x_{26} & +  0.02 x_{28} & -1.24 x_{7} & +  0.08 x_{32} & +  0.03 x_{20} & +  0.07 x_{27} & -1.60 x_{4} & -1.08 x_{8} & -1.45 x_{10} & +  0.02 x_{2} & -1.39 x_{12} & -0.49 x_{13} & -1.49 x_{3} & -0.02 x_{25} & +  0.05 x_{24} & +  1.01 x_{17}\\
\hline
z    &  4.94777989437 & -5.04 x_{5} & +  0.78 x_{26} & +  0.48 x_{28} & -22.24 x_{7} & +  0.50 x_{32} & +  0.66 x_{20} & +  0.47 x_{27} & -16.19 x_{4} & -11.53 x_{8} & -15.27 x_{10} & +  2.57 x_{2} & -10.57 x_{12} & -8.49 x_{13} & -8.76 x_{3} & -0.46 x_{25} & -0.19 x_{24} & +  6.01 x_{17}\\
\end{array}\]


 $ x_{2} $ enters and $ x_{11} $ leaves 

 \[\begin{array}{c| c c@{\hskip 2pt} c@{\hskip 2pt} c@{\hskip 2pt} c@{\hskip 2pt} c@{\hskip 2pt} c@{\hskip 2pt} c@{\hskip 2pt} c@{\hskip 2pt} c@{\hskip 2pt} c@{\hskip 2pt} c@{\hskip 2pt} c@{\hskip 2pt} c@{\hskip 2pt} c@{\hskip 2pt} c@{\hskip 2pt} c@{\hskip 2pt} c@{\hskip 2pt} }
 x_{18}   &  7.19897450558 & -4.31 x_{5} & +  0.64 x_{26} & +  0.62 x_{28} & -11.16 x_{7} & +  0.31 x_{32} & +  0.06 x_{20} & +  0.45 x_{27} & -14.29 x_{4} & -15.02 x_{8} & -14.34 x_{10} & -0.90 x_{11} & -12.21 x_{12} & -10.72 x_{13} & -1.90 x_{3} & -0.03 x_{25} & -0.21 x_{24} & -8.59 x_{17}\\
 x_{1}   &  0.727420747408 & -1.03 x_{5} & +  0.11 x_{26} & +  0.04 x_{28} & -3.06 x_{7} & +  0.05 x_{32} & +  0.10 x_{20} & +  0.11 x_{27} & -2.26 x_{4} & -2.39 x_{8} & -1.94 x_{10} & -0.78 x_{11} & -2.22 x_{12} & -1.68 x_{13} & -1.48 x_{3} & -0.05 x_{25} & -0.04 x_{24} & +  1.40 x_{17}\\
 x_{19}   &  7.0819250477 & -2.65 x_{5} & +  0.77 x_{26} & +  0.76 x_{28} & -21.24 x_{7} & -0.14 x_{32} & +  1.16 x_{20} & +  1.17 x_{27} & -9.56 x_{4} & -20.65 x_{8} & -6.00 x_{10} & -9.33 x_{11} & -6.84 x_{12} & -20.12 x_{13} & -10.28 x_{3} & -0.22 x_{25} & -0.52 x_{24} & +  3.97 x_{17}\\
 x_{21}   &  2.8991490619 & -11.18 x_{5} & +  0.63 x_{26} & -0.16 x_{28} & +  7.63 x_{7} & +  1.09 x_{32} & -0.53 x_{20} & -0.29 x_{27} & -7.17 x_{4} & +  3.02 x_{8} & -16.10 x_{10} & + 21.88 x_{11} & -9.62 x_{12} & + 18.01 x_{13} & -18.05 x_{3} & +  0.60 x_{25} & +  0.99 x_{24} & -10.53 x_{17}\\
 x_{22}   &  8.81902483758 & -1.68 x_{5} & +  1.51 x_{26} & +  0.50 x_{28} & -29.80 x_{7} & +  1.33 x_{32} & +  0.71 x_{20} & +  1.10 x_{27} & -31.63 x_{4} & -24.22 x_{8} & -23.66 x_{10} & +  5.27 x_{11} & -32.80 x_{12} & -3.91 x_{13} & -33.04 x_{3} & -0.23 x_{25} & +  0.16 x_{24} & + 19.45 x_{17}\\
 x_{23}   &  8.55155504579 & -10.14 x_{5} & +  1.66 x_{26} & +  1.26 x_{28} & -13.64 x_{7} & +  1.38 x_{32} & +  0.25 x_{20} & +  0.69 x_{27} & -39.33 x_{4} & -30.42 x_{8} & -28.68 x_{10} & -4.98 x_{11} & -18.12 x_{12} & -18.43 x_{13} & -36.04 x_{3} & -0.27 x_{25} & +  0.41 x_{24} & -1.66 x_{17}\\
 x_{16}   &  0.207713818077 & +  0.15 x_{5} & -0.01 x_{26} & +  0.03 x_{28} & -1.54 x_{7} & -0.03 x_{32} & +  0.09 x_{20} & +  0.07 x_{27} & -0.66 x_{4} & -0.71 x_{8} & -0.38 x_{10} & -1.06 x_{11} & -0.77 x_{12} & -1.72 x_{13} & +  0.71 x_{3} & -0.06 x_{25} & -0.04 x_{24} & +  0.92 x_{17}\\
 x_{15}   &  0.358152239935 & -0.19 x_{5} & +  0.07 x_{26} & +  0.03 x_{28} & -1.04 x_{7} & +  0.05 x_{32} & +  0.04 x_{20} & +  0.02 x_{27} & -1.46 x_{4} & -0.68 x_{8} & -1.14 x_{10} & +  0.17 x_{11} & -0.51 x_{12} & -0.38 x_{13} & -0.86 x_{3} & +  0.01 x_{25} & +  0.04 x_{24} & +  0.50 x_{17}\\
 x_{14}   &  0.434600166394 & -0.19 x_{5} & +  0.05 x_{26} & +  0.06 x_{28} & -1.33 x_{7} & +  0.07 x_{32} & +  0.04 x_{20} & +  0.02 x_{27} & -1.89 x_{4} & -1.57 x_{8} & -1.38 x_{10} & -0.06 x_{11} & -0.67 x_{12} & -0.47 x_{13} & -0.83 x_{3} & -0.04 x_{25} & -0.03 x_{24} & +  0.27 x_{17}\\
 x_{6}   &  0.257564650027 & -0.42 x_{5} & -0.01 x_{26} & -0.02 x_{28} & -0.60 x_{7} & +  0.02 x_{32} & +  0.06 x_{20} & +  0.00 x_{27} & -0.37 x_{4} & -0.21 x_{8} & -0.08 x_{10} & +  0.65 x_{11} & -0.29 x_{12} & -0.24 x_{13} & +  0.05 x_{3} & +  0.02 x_{25} & -0.02 x_{24} & +  0.22 x_{17}\\
 x_{2}   &  0.112843781101 & +  0.45 x_{5} & -0.02 x_{26} & +  0.06 x_{28} & -0.22 x_{7} & -0.02 x_{32} & +  0.00 x_{20} & +  0.05 x_{27} & -0.22 x_{4} & -0.27 x_{8} & +  0.64 x_{10} & -0.97 x_{11} & +  0.26 x_{12} & -0.96 x_{13} & +  0.27 x_{3} & +  0.01 x_{25} & -0.01 x_{24} & -0.37 x_{17}\\
 x_{29}   &  4.91704225336 & -9.20 x_{5} & +  0.98 x_{26} & -0.25 x_{28} & -11.44 x_{7} & +  0.52 x_{32} & +  0.44 x_{20} & +  0.08 x_{27} & -14.95 x_{4} & -1.42 x_{8} & -14.95 x_{10} & +  7.26 x_{11} & -14.56 x_{12} & +  7.12 x_{13} & -21.77 x_{3} & +  0.23 x_{25} & +  0.02 x_{24} & +  8.02 x_{17}\\
 x_{30}   &  3.42692206434 & +  1.42 x_{5} & +  0.37 x_{26} & +  0.06 x_{28} & +  0.65 x_{7} & -0.15 x_{32} & -0.55 x_{20} & -0.04 x_{27} & +  0.73 x_{4} & +  9.87 x_{8} & + 12.71 x_{10} & -6.77 x_{11} & +  4.44 x_{12} & -2.98 x_{13} & -13.78 x_{3} & +  0.83 x_{25} & +  0.49 x_{24} & -3.97 x_{17}\\
 x_{31}   &  7.07185734438 & -13.42 x_{5} & +  1.06 x_{26} & +  0.06 x_{28} & -1.29 x_{7} & +  1.34 x_{32} & -0.08 x_{20} & -0.45 x_{27} & -14.20 x_{4} & -3.05 x_{8} & -8.24 x_{10} & + 13.72 x_{11} & +  5.95 x_{12} & + 16.05 x_{13} & -23.89 x_{3} & +  0.52 x_{25} & +  0.62 x_{24} & -5.98 x_{17}\\
 x_{9}   &  0.205618145886 & -0.15 x_{5} & +  0.07 x_{26} & +  0.02 x_{28} & -1.24 x_{7} & +  0.08 x_{32} & +  0.03 x_{20} & +  0.07 x_{27} & -1.61 x_{4} & -1.09 x_{8} & -1.44 x_{10} & -0.02 x_{11} & -1.39 x_{12} & -0.51 x_{13} & -1.48 x_{3} & -0.02 x_{25} & +  0.05 x_{24} & +  1.00 x_{17}\\
\hline
z    &  5.23724376531 & -3.89 x_{5} & +  0.73 x_{26} & +  0.63 x_{28} & -22.79 x_{7} & +  0.45 x_{32} & +  0.66 x_{20} & +  0.59 x_{27} & -16.76 x_{4} & -12.23 x_{8} & -13.63 x_{10} & -2.50 x_{11} & -9.89 x_{12} & -10.96 x_{13} & -8.07 x_{3} & -0.44 x_{25} & -0.23 x_{24} & +  5.05 x_{17}\\
\end{array}\]


 $ x_{17} $ enters and $ x_{21} $ leaves 

 \[\begin{array}{c| c c@{\hskip 2pt} c@{\hskip 2pt} c@{\hskip 2pt} c@{\hskip 2pt} c@{\hskip 2pt} c@{\hskip 2pt} c@{\hskip 2pt} c@{\hskip 2pt} c@{\hskip 2pt} c@{\hskip 2pt} c@{\hskip 2pt} c@{\hskip 2pt} c@{\hskip 2pt} c@{\hskip 2pt} c@{\hskip 2pt} c@{\hskip 2pt} c@{\hskip 2pt} }
 x_{18}   &  4.83513165043 & +  4.81 x_{5} & +  0.13 x_{26} & +  0.75 x_{28} & -17.38 x_{7} & -0.58 x_{32} & +  0.50 x_{20} & +  0.69 x_{27} & -8.45 x_{4} & -17.48 x_{8} & -1.21 x_{10} & -18.73 x_{11} & -4.36 x_{12} & -25.40 x_{13} & + 12.82 x_{3} & -0.53 x_{25} & -1.02 x_{24} & +  0.82 x_{21}\\
 x_{1}   &  1.11312467269 & -2.52 x_{5} & +  0.20 x_{26} & +  0.02 x_{28} & -2.04 x_{7} & +  0.20 x_{32} & +  0.03 x_{20} & +  0.07 x_{27} & -3.21 x_{4} & -1.98 x_{8} & -4.08 x_{10} & +  2.13 x_{11} & -3.50 x_{12} & +  0.71 x_{13} & -3.88 x_{3} & +  0.03 x_{25} & +  0.09 x_{24} & -0.13 x_{21}\\
 x_{19}   &  8.17401818311 & -6.86 x_{5} & +  1.00 x_{26} & +  0.70 x_{28} & -18.36 x_{7} & +  0.28 x_{32} & +  0.96 x_{20} & +  1.07 x_{27} & -12.26 x_{4} & -19.52 x_{8} & -12.07 x_{10} & -1.09 x_{11} & -10.47 x_{12} & -13.33 x_{13} & -17.08 x_{3} & +  0.01 x_{25} & -0.15 x_{24} & -0.38 x_{21}\\
 x_{17}   &  0.275322846898 & -1.06 x_{5} & +  0.06 x_{26} & -0.01 x_{28} & +  0.73 x_{7} & +  0.10 x_{32} & -0.05 x_{20} & -0.03 x_{27} & -0.68 x_{4} & +  0.29 x_{8} & -1.53 x_{10} & +  2.08 x_{11} & -0.91 x_{12} & +  1.71 x_{13} & -1.71 x_{3} & +  0.06 x_{25} & +  0.09 x_{24} & -0.09 x_{21}\\
 x_{22}   &  14.1750983372 & -22.33 x_{5} & +  2.67 x_{26} & +  0.21 x_{28} & -15.69 x_{7} & +  3.34 x_{32} & -0.27 x_{20} & +  0.57 x_{27} & -44.88 x_{4} & -18.65 x_{8} & -53.41 x_{10} & + 45.69 x_{11} & -50.58 x_{12} & + 29.36 x_{13} & -66.39 x_{3} & +  0.88 x_{25} & +  1.99 x_{24} & -1.85 x_{21}\\
 x_{23}   &  8.09516361419 & -8.38 x_{5} & +  1.56 x_{26} & +  1.28 x_{28} & -14.84 x_{7} & +  1.21 x_{32} & +  0.33 x_{20} & +  0.74 x_{27} & -38.20 x_{4} & -30.89 x_{8} & -26.14 x_{10} & -8.42 x_{11} & -16.61 x_{12} & -21.26 x_{13} & -33.20 x_{3} & -0.36 x_{25} & +  0.26 x_{24} & +  0.16 x_{21}\\
 x_{16}   &  0.461822643399 & -0.83 x_{5} & +  0.05 x_{26} & +  0.02 x_{28} & -0.87 x_{7} & +  0.06 x_{32} & +  0.04 x_{20} & +  0.05 x_{27} & -1.29 x_{4} & -0.44 x_{8} & -1.79 x_{10} & +  0.86 x_{11} & -1.62 x_{12} & -0.14 x_{13} & -0.87 x_{3} & -0.00 x_{25} & +  0.05 x_{24} & -0.09 x_{21}\\
 x_{15}   &  0.496030231126 & -0.72 x_{5} & +  0.10 x_{26} & +  0.02 x_{28} & -0.68 x_{7} & +  0.10 x_{32} & +  0.02 x_{20} & +  0.01 x_{27} & -1.80 x_{4} & -0.54 x_{8} & -1.90 x_{10} & +  1.21 x_{11} & -0.97 x_{12} & +  0.47 x_{13} & -1.72 x_{3} & +  0.04 x_{25} & +  0.09 x_{24} & -0.05 x_{21}\\
 x_{14}   &  0.507647428633 & -0.47 x_{5} & +  0.07 x_{26} & +  0.05 x_{28} & -1.14 x_{7} & +  0.10 x_{32} & +  0.03 x_{20} & +  0.01 x_{27} & -2.07 x_{4} & -1.49 x_{8} & -1.78 x_{10} & +  0.49 x_{11} & -0.92 x_{12} & -0.02 x_{13} & -1.28 x_{3} & -0.02 x_{25} & -0.00 x_{24} & -0.03 x_{21}\\
 x_{6}   &  0.319272740342 & -0.65 x_{5} & +  0.01 x_{26} & -0.03 x_{28} & -0.44 x_{7} & +  0.04 x_{32} & +  0.05 x_{20} & -0.00 x_{27} & -0.53 x_{4} & -0.14 x_{8} & -0.42 x_{10} & +  1.11 x_{11} & -0.50 x_{12} & +  0.14 x_{13} & -0.34 x_{3} & +  0.03 x_{25} & +  0.00 x_{24} & -0.02 x_{21}\\
 x_{2}   &  0.0106937593112 & +  0.84 x_{5} & -0.04 x_{26} & +  0.06 x_{28} & -0.48 x_{7} & -0.06 x_{32} & +  0.02 x_{20} & +  0.06 x_{27} & +  0.03 x_{4} & -0.38 x_{8} & +  1.21 x_{10} & -1.75 x_{11} & +  0.60 x_{12} & -1.60 x_{13} & +  0.90 x_{3} & -0.01 x_{25} & -0.05 x_{24} & +  0.04 x_{21}\\
 x_{29}   &  7.12549610238 & -17.71 x_{5} & +  1.46 x_{26} & -0.37 x_{28} & -5.63 x_{7} & +  1.35 x_{32} & +  0.04 x_{20} & -0.14 x_{27} & -20.42 x_{4} & +  0.87 x_{8} & -27.21 x_{10} & + 23.92 x_{11} & -21.89 x_{12} & + 20.84 x_{13} & -35.52 x_{3} & +  0.69 x_{25} & +  0.77 x_{24} & -0.76 x_{21}\\
 x_{30}   &  2.33433491477 & +  5.63 x_{5} & +  0.13 x_{26} & +  0.11 x_{28} & -2.23 x_{7} & -0.56 x_{32} & -0.35 x_{20} & +  0.07 x_{27} & +  3.43 x_{4} & +  8.73 x_{8} & + 18.78 x_{10} & -15.02 x_{11} & +  8.07 x_{12} & -9.76 x_{13} & -6.98 x_{3} & +  0.60 x_{25} & +  0.12 x_{24} & +  0.38 x_{21}\\
 x_{31}   &  5.42597260653 & -7.08 x_{5} & +  0.70 x_{26} & +  0.15 x_{28} & -5.62 x_{7} & +  0.72 x_{32} & +  0.22 x_{20} & -0.28 x_{27} & -10.13 x_{4} & -4.76 x_{8} & +  0.90 x_{10} & +  1.30 x_{11} & + 11.42 x_{12} & +  5.82 x_{13} & -13.64 x_{3} & +  0.17 x_{25} & +  0.06 x_{24} & +  0.57 x_{21}\\
 x_{9}   &  0.481873349305 & -1.21 x_{5} & +  0.13 x_{26} & +  0.00 x_{28} & -0.52 x_{7} & +  0.18 x_{32} & -0.02 x_{20} & +  0.04 x_{27} & -2.29 x_{4} & -0.80 x_{8} & -2.98 x_{10} & +  2.07 x_{11} & -2.30 x_{12} & +  1.20 x_{13} & -3.20 x_{3} & +  0.04 x_{25} & +  0.14 x_{24} & -0.10 x_{21}\\
\hline
z    &  6.62897637695 & -9.25 x_{5} & +  1.03 x_{26} & +  0.55 x_{28} & -19.13 x_{7} & +  0.98 x_{32} & +  0.40 x_{20} & +  0.45 x_{27} & -20.20 x_{4} & -10.78 x_{8} & -21.36 x_{10} & +  8.00 x_{11} & -14.51 x_{12} & -2.32 x_{13} & -16.73 x_{3} & -0.15 x_{25} & +  0.25 x_{24} & -0.48 x_{21}\\
\end{array}\]


 $ x_{11} $ enters and $ x_{2} $ leaves 

 \[\begin{array}{c| c c@{\hskip 2pt} c@{\hskip 2pt} c@{\hskip 2pt} c@{\hskip 2pt} c@{\hskip 2pt} c@{\hskip 2pt} c@{\hskip 2pt} c@{\hskip 2pt} c@{\hskip 2pt} c@{\hskip 2pt} c@{\hskip 2pt} c@{\hskip 2pt} c@{\hskip 2pt} c@{\hskip 2pt} c@{\hskip 2pt} c@{\hskip 2pt} c@{\hskip 2pt} }
 x_{18}   &  4.72037993544 & -4.25 x_{5} & +  0.58 x_{26} & +  0.06 x_{28} & -12.18 x_{7} & +  0.03 x_{32} & +  0.29 x_{20} & +  0.06 x_{27} & -8.79 x_{4} & -13.43 x_{8} & -14.16 x_{10} & + 10.73 x_{2} & -10.84 x_{12} & -8.26 x_{13} & +  3.12 x_{3} & -0.39 x_{25} & -0.48 x_{24} & +  0.44 x_{21}\\
 x_{1}   &  1.12617680586 & -1.49 x_{5} & +  0.14 x_{26} & +  0.10 x_{28} & -2.63 x_{7} & +  0.13 x_{32} & +  0.05 x_{20} & +  0.14 x_{27} & -3.17 x_{4} & -2.45 x_{8} & -2.61 x_{10} & -1.22 x_{2} & -2.76 x_{12} & -1.24 x_{13} & -2.78 x_{3} & +  0.01 x_{25} & +  0.03 x_{24} & -0.09 x_{21}\\
 x_{19}   &  8.16733712652 & -7.39 x_{5} & +  1.03 x_{26} & +  0.66 x_{28} & -18.06 x_{7} & +  0.31 x_{32} & +  0.95 x_{20} & +  1.03 x_{27} & -12.28 x_{4} & -19.28 x_{8} & -12.82 x_{10} & +  0.62 x_{2} & -10.84 x_{12} & -12.34 x_{13} & -17.64 x_{3} & +  0.02 x_{25} & -0.12 x_{24} & -0.40 x_{21}\\
 x_{17}   &  0.288049156002 & -0.06 x_{5} & +  0.01 x_{26} & +  0.06 x_{28} & +  0.15 x_{7} & +  0.04 x_{32} & -0.03 x_{20} & +  0.04 x_{27} & -0.64 x_{4} & -0.16 x_{8} & -0.09 x_{10} & -1.19 x_{2} & -0.20 x_{12} & -0.19 x_{13} & -0.64 x_{3} & +  0.04 x_{25} & +  0.03 x_{24} & -0.05 x_{21}\\
 x_{22}   &  14.4549799516 & -0.25 x_{5} & +  1.58 x_{26} & +  1.90 x_{28} & -28.37 x_{7} & +  1.86 x_{32} & +  0.23 x_{20} & +  2.09 x_{27} & -44.04 x_{4} & -28.52 x_{8} & -21.81 x_{10} & -26.17 x_{2} & -34.79 x_{12} & -12.44 x_{13} & -42.73 x_{3} & +  0.55 x_{25} & +  0.69 x_{24} & -0.93 x_{21}\\
 x_{23}   &  8.04358845004 & -12.45 x_{5} & +  1.76 x_{26} & +  0.97 x_{28} & -12.50 x_{7} & +  1.49 x_{32} & +  0.24 x_{20} & +  0.46 x_{27} & -38.36 x_{4} & -29.07 x_{8} & -31.97 x_{10} & +  4.82 x_{2} & -19.52 x_{12} & -13.56 x_{13} & -37.56 x_{3} & -0.30 x_{25} & +  0.50 x_{24} & -0.01 x_{21}\\
 x_{16}   &  0.467076533911 & -0.42 x_{5} & +  0.03 x_{26} & +  0.05 x_{28} & -1.10 x_{7} & +  0.03 x_{32} & +  0.05 x_{20} & +  0.07 x_{27} & -1.27 x_{4} & -0.63 x_{8} & -1.20 x_{10} & -0.49 x_{2} & -1.32 x_{12} & -0.93 x_{13} & -0.42 x_{3} & -0.01 x_{25} & +  0.03 x_{24} & -0.07 x_{21}\\
 x_{15}   &  0.503453953402 & -0.14 x_{5} & +  0.07 x_{26} & +  0.06 x_{28} & -1.01 x_{7} & +  0.06 x_{32} & +  0.03 x_{20} & +  0.05 x_{27} & -1.77 x_{4} & -0.80 x_{8} & -1.06 x_{10} & -0.69 x_{2} & -0.55 x_{12} & -0.64 x_{13} & -1.09 x_{3} & +  0.03 x_{25} & +  0.05 x_{24} & -0.02 x_{21}\\
 x_{14}   &  0.510652369949 & -0.23 x_{5} & +  0.06 x_{26} & +  0.07 x_{28} & -1.28 x_{7} & +  0.08 x_{32} & +  0.03 x_{20} & +  0.03 x_{27} & -2.06 x_{4} & -1.60 x_{8} & -1.44 x_{10} & -0.28 x_{2} & -0.75 x_{12} & -0.47 x_{13} & -1.03 x_{3} & -0.03 x_{25} & -0.02 x_{24} & -0.02 x_{21}\\
 x_{6}   &  0.326099834815 & -0.12 x_{5} & -0.02 x_{26} & +  0.02 x_{28} & -0.75 x_{7} & +  0.01 x_{32} & +  0.06 x_{20} & +  0.03 x_{27} & -0.51 x_{4} & -0.38 x_{8} & +  0.35 x_{10} & -0.64 x_{2} & -0.11 x_{12} & -0.88 x_{13} & +  0.24 x_{3} & +  0.03 x_{25} & -0.03 x_{24} & +  0.00 x_{21}\\
 x_{11}   &  0.00612543854601 & +  0.48 x_{5} & -0.02 x_{26} & +  0.04 x_{28} & -0.28 x_{7} & -0.03 x_{32} & +  0.01 x_{20} & +  0.03 x_{27} & +  0.02 x_{4} & -0.22 x_{8} & +  0.69 x_{10} & -0.57 x_{2} & +  0.35 x_{12} & -0.91 x_{13} & +  0.52 x_{3} & -0.01 x_{25} & -0.03 x_{24} & +  0.02 x_{21}\\
 x_{29}   &  7.27202532951 & -6.15 x_{5} & +  0.88 x_{26} & +  0.51 x_{28} & -12.27 x_{7} & +  0.58 x_{32} & +  0.30 x_{20} & +  0.66 x_{27} & -19.98 x_{4} & -4.30 x_{8} & -10.67 x_{10} & -13.70 x_{2} & -13.63 x_{12} & -1.04 x_{13} & -23.13 x_{3} & +  0.51 x_{25} & +  0.09 x_{24} & -0.28 x_{21}\\
 x_{30}   &  2.24234083474 & -1.63 x_{5} & +  0.49 x_{26} & -0.44 x_{28} & +  1.94 x_{7} & -0.07 x_{32} & -0.52 x_{20} & -0.44 x_{27} & +  3.15 x_{4} & + 11.98 x_{8} & +  8.39 x_{10} & +  8.60 x_{2} & +  2.88 x_{12} & +  3.97 x_{13} & -14.76 x_{3} & +  0.71 x_{25} & +  0.55 x_{24} & +  0.07 x_{21}\\
 x_{31}   &  5.43390852382 & -6.45 x_{5} & +  0.67 x_{26} & +  0.20 x_{28} & -5.98 x_{7} & +  0.68 x_{32} & +  0.23 x_{20} & -0.24 x_{27} & -10.11 x_{4} & -5.04 x_{8} & +  1.80 x_{10} & -0.74 x_{2} & + 11.86 x_{12} & +  4.64 x_{13} & -12.97 x_{3} & +  0.16 x_{25} & +  0.02 x_{24} & +  0.59 x_{21}\\
 x_{9}   &  0.494523585222 & -0.21 x_{5} & +  0.08 x_{26} & +  0.08 x_{28} & -1.09 x_{7} & +  0.12 x_{32} & -0.00 x_{20} & +  0.11 x_{27} & -2.25 x_{4} & -1.25 x_{8} & -1.55 x_{10} & -1.18 x_{2} & -1.59 x_{12} & -0.69 x_{13} & -2.13 x_{3} & +  0.02 x_{25} & +  0.08 x_{24} & -0.05 x_{21}\\
\hline
z    &  6.67798738267 & -5.38 x_{5} & +  0.84 x_{26} & +  0.85 x_{28} & -21.35 x_{7} & +  0.72 x_{32} & +  0.49 x_{20} & +  0.72 x_{27} & -20.06 x_{4} & -12.51 x_{8} & -15.83 x_{10} & -4.58 x_{2} & -11.74 x_{12} & -9.64 x_{13} & -12.59 x_{3} & -0.21 x_{25} & +  0.02 x_{24} & -0.32 x_{21}\\
\end{array}\]


 $ x_{20} $ enters and $ x_{30} $ leaves 

 \[\begin{array}{c| c c@{\hskip 2pt} c@{\hskip 2pt} c@{\hskip 2pt} c@{\hskip 2pt} c@{\hskip 2pt} c@{\hskip 2pt} c@{\hskip 2pt} c@{\hskip 2pt} c@{\hskip 2pt} c@{\hskip 2pt} c@{\hskip 2pt} c@{\hskip 2pt} c@{\hskip 2pt} c@{\hskip 2pt} c@{\hskip 2pt} c@{\hskip 2pt} c@{\hskip 2pt} }
 x_{18}   &  5.96910254896 & -5.15 x_{5} & +  0.86 x_{26} & -0.19 x_{28} & -11.10 x_{7} & -0.01 x_{32} & -0.56 x_{30} & -0.18 x_{27} & -7.03 x_{4} & -6.76 x_{8} & -9.49 x_{10} & + 15.52 x_{2} & -9.23 x_{12} & -6.05 x_{13} & -5.10 x_{3} & +  0.00 x_{25} & -0.18 x_{24} & +  0.48 x_{21}\\
 x_{1}   &  1.33992112292 & -1.65 x_{5} & +  0.19 x_{26} & +  0.06 x_{28} & -2.45 x_{7} & +  0.12 x_{32} & -0.10 x_{30} & +  0.10 x_{27} & -2.87 x_{4} & -1.30 x_{8} & -1.81 x_{10} & -0.40 x_{2} & -2.49 x_{12} & -0.86 x_{13} & -4.18 x_{3} & +  0.08 x_{25} & +  0.08 x_{24} & -0.08 x_{21}\\
 x_{19}   &  12.2913490457 & -10.38 x_{5} & +  1.93 x_{26} & -0.15 x_{28} & -14.49 x_{7} & +  0.18 x_{32} & -1.84 x_{30} & +  0.23 x_{27} & -6.48 x_{4} & +  2.75 x_{8} & +  2.61 x_{10} & + 16.45 x_{2} & -5.55 x_{12} & -5.03 x_{13} & -44.79 x_{3} & +  1.32 x_{25} & +  0.89 x_{24} & -0.26 x_{21}\\
 x_{17}   &  0.168467793204 & +  0.03 x_{5} & -0.02 x_{26} & +  0.09 x_{28} & +  0.05 x_{7} & +  0.04 x_{32} & +  0.05 x_{30} & +  0.07 x_{27} & -0.81 x_{4} & -0.80 x_{8} & -0.54 x_{10} & -1.65 x_{2} & -0.35 x_{12} & -0.40 x_{13} & +  0.15 x_{3} & +  0.00 x_{25} & +  0.01 x_{24} & -0.06 x_{21}\\
 x_{22}   &  15.4660734857 & -0.98 x_{5} & +  1.80 x_{26} & +  1.70 x_{28} & -27.49 x_{7} & +  1.82 x_{32} & -0.45 x_{30} & +  1.90 x_{27} & -42.62 x_{4} & -23.12 x_{8} & -18.02 x_{10} & -22.29 x_{2} & -33.49 x_{12} & -10.64 x_{13} & -49.38 x_{3} & +  0.87 x_{25} & +  0.93 x_{24} & -0.89 x_{21}\\
 x_{23}   &  9.06707022066 & -13.19 x_{5} & +  1.98 x_{26} & +  0.77 x_{28} & -11.62 x_{7} & +  1.45 x_{32} & -0.46 x_{30} & +  0.26 x_{27} & -36.92 x_{4} & -23.60 x_{8} & -28.14 x_{10} & +  8.75 x_{2} & -18.20 x_{12} & -11.75 x_{13} & -44.30 x_{3} & +  0.02 x_{25} & +  0.75 x_{24} & +  0.02 x_{21}\\
 x_{16}   &  0.683966087351 & -0.58 x_{5} & +  0.08 x_{26} & +  0.00 x_{28} & -0.92 x_{7} & +  0.03 x_{32} & -0.10 x_{30} & +  0.03 x_{27} & -0.97 x_{4} & +  0.53 x_{8} & -0.39 x_{10} & +  0.34 x_{2} & -1.04 x_{12} & -0.54 x_{13} & -1.85 x_{3} & +  0.06 x_{25} & +  0.08 x_{24} & -0.06 x_{21}\\
 x_{15}   &  0.636207132316 & -0.23 x_{5} & +  0.10 x_{26} & +  0.04 x_{28} & -0.90 x_{7} & +  0.06 x_{32} & -0.06 x_{30} & +  0.02 x_{27} & -1.59 x_{4} & -0.09 x_{8} & -0.57 x_{10} & -0.18 x_{2} & -0.38 x_{12} & -0.40 x_{13} & -1.96 x_{3} & +  0.07 x_{25} & +  0.09 x_{24} & -0.02 x_{21}\\
 x_{14}   &  0.655205246571 & -0.34 x_{5} & +  0.09 x_{26} & +  0.04 x_{28} & -1.15 x_{7} & +  0.08 x_{32} & -0.06 x_{30} & -0.00 x_{27} & -1.86 x_{4} & -0.82 x_{8} & -0.90 x_{10} & +  0.27 x_{2} & -0.56 x_{12} & -0.21 x_{13} & -1.98 x_{3} & +  0.02 x_{25} & +  0.02 x_{24} & -0.01 x_{21}\\
 x_{6}   &  0.580706954416 & -0.30 x_{5} & +  0.04 x_{26} & -0.03 x_{28} & -0.53 x_{7} & -0.00 x_{32} & -0.11 x_{30} & -0.01 x_{27} & -0.15 x_{4} & +  0.98 x_{8} & +  1.30 x_{10} & +  0.34 x_{2} & +  0.22 x_{12} & -0.43 x_{13} & -1.44 x_{3} & +  0.11 x_{25} & +  0.03 x_{24} & +  0.01 x_{21}\\
 x_{11}   &  0.0541557467288 & +  0.45 x_{5} & -0.01 x_{26} & +  0.03 x_{28} & -0.24 x_{7} & -0.03 x_{32} & -0.02 x_{30} & +  0.02 x_{27} & +  0.09 x_{4} & +  0.04 x_{8} & +  0.87 x_{10} & -0.39 x_{2} & +  0.41 x_{12} & -0.83 x_{13} & +  0.20 x_{3} & +  0.01 x_{25} & -0.02 x_{24} & +  0.02 x_{21}\\
 x_{29}   &  8.58826958834 & -7.11 x_{5} & +  1.17 x_{26} & +  0.25 x_{28} & -11.13 x_{7} & +  0.53 x_{32} & -0.59 x_{30} & +  0.40 x_{27} & -18.13 x_{4} & +  2.74 x_{8} & -5.75 x_{10} & -8.65 x_{2} & -11.94 x_{12} & +  1.29 x_{13} & -31.80 x_{3} & +  0.93 x_{25} & +  0.41 x_{24} & -0.24 x_{21}\\
 x_{20}   &  4.33559159055 & -3.15 x_{5} & +  0.95 x_{26} & -0.85 x_{28} & +  3.75 x_{7} & -0.14 x_{32} & -1.93 x_{30} & -0.84 x_{27} & +  6.10 x_{4} & + 23.16 x_{8} & + 16.22 x_{10} & + 16.63 x_{2} & +  5.56 x_{12} & +  7.68 x_{13} & -28.53 x_{3} & +  1.37 x_{25} & +  1.06 x_{24} & +  0.14 x_{21}\\
 x_{31}   &  6.44319654764 & -7.18 x_{5} & +  0.90 x_{26} & -0.00 x_{28} & -5.11 x_{7} & +  0.65 x_{32} & -0.45 x_{30} & -0.44 x_{27} & -8.69 x_{4} & +  0.35 x_{8} & +  5.57 x_{10} & +  3.13 x_{2} & + 13.16 x_{12} & +  6.43 x_{13} & -19.61 x_{3} & +  0.48 x_{25} & +  0.27 x_{24} & +  0.63 x_{21}\\
 x_{9}   &  0.48949659369 & -0.21 x_{5} & +  0.08 x_{26} & +  0.08 x_{28} & -1.09 x_{7} & +  0.12 x_{32} & +  0.00 x_{30} & +  0.11 x_{27} & -2.26 x_{4} & -1.27 x_{8} & -1.57 x_{10} & -1.20 x_{2} & -1.60 x_{12} & -0.70 x_{13} & -2.10 x_{3} & +  0.02 x_{25} & +  0.08 x_{24} & -0.05 x_{21}\\
\hline
z    &  8.80112007125 & -6.93 x_{5} & +  1.31 x_{26} & +  0.43 x_{28} & -19.51 x_{7} & +  0.65 x_{32} & -0.95 x_{30} & +  0.31 x_{27} & -17.07 x_{4} & -1.17 x_{8} & -7.89 x_{10} & +  3.56 x_{2} & -9.02 x_{12} & -5.88 x_{13} & -26.56 x_{3} & +  0.46 x_{25} & +  0.53 x_{24} & -0.25 x_{21}\\
\end{array}\]


 $ x_{2} $ enters and $ x_{17} $ leaves 

 \[\begin{array}{c| c c@{\hskip 2pt} c@{\hskip 2pt} c@{\hskip 2pt} c@{\hskip 2pt} c@{\hskip 2pt} c@{\hskip 2pt} c@{\hskip 2pt} c@{\hskip 2pt} c@{\hskip 2pt} c@{\hskip 2pt} c@{\hskip 2pt} c@{\hskip 2pt} c@{\hskip 2pt} c@{\hskip 2pt} c@{\hskip 2pt} c@{\hskip 2pt} c@{\hskip 2pt} }
 x_{18}   &  7.55497960761 & -4.88 x_{5} & +  0.70 x_{26} & +  0.62 x_{28} & -10.68 x_{7} & +  0.37 x_{32} & -0.05 x_{30} & +  0.43 x_{27} & -14.67 x_{4} & -14.30 x_{8} & -14.57 x_{10} & -9.41 x_{17} & -12.52 x_{12} & -9.84 x_{13} & -3.70 x_{3} & +  0.05 x_{25} & -0.12 x_{24} & -0.06 x_{21}\\
 x_{1}   &  1.29899833334 & -1.65 x_{5} & +  0.20 x_{26} & +  0.03 x_{28} & -2.46 x_{7} & +  0.11 x_{32} & -0.11 x_{30} & +  0.09 x_{27} & -2.67 x_{4} & -1.11 x_{8} & -1.68 x_{10} & +  0.24 x_{17} & -2.40 x_{12} & -0.76 x_{13} & -4.22 x_{3} & +  0.08 x_{25} & +  0.08 x_{24} & -0.07 x_{21}\\
 x_{19}   &  13.9717271902 & -10.09 x_{5} & +  1.77 x_{26} & +  0.70 x_{28} & -14.04 x_{7} & +  0.58 x_{32} & -1.31 x_{30} & +  0.88 x_{27} & -14.57 x_{4} & -5.25 x_{8} & -2.78 x_{10} & -9.97 x_{17} & -9.03 x_{12} & -9.04 x_{13} & -43.30 x_{3} & +  1.37 x_{25} & +  0.95 x_{24} & -0.83 x_{21}\\
 x_{2}   &  0.102173839229 & +  0.02 x_{5} & -0.01 x_{26} & +  0.05 x_{28} & +  0.03 x_{7} & +  0.02 x_{32} & +  0.03 x_{30} & +  0.04 x_{27} & -0.49 x_{4} & -0.49 x_{8} & -0.33 x_{10} & -0.61 x_{17} & -0.21 x_{12} & -0.24 x_{13} & +  0.09 x_{3} & +  0.00 x_{25} & +  0.00 x_{24} & -0.03 x_{21}\\
 x_{22}   &  13.1882681111 & -1.38 x_{5} & +  2.02 x_{26} & +  0.54 x_{28} & -28.10 x_{7} & +  1.28 x_{32} & -1.17 x_{30} & +  1.01 x_{27} & -31.65 x_{4} & -12.29 x_{8} & -10.72 x_{10} & + 13.52 x_{17} & -28.77 x_{12} & -5.20 x_{13} & -51.39 x_{3} & +  0.81 x_{25} & +  0.86 x_{24} & -0.12 x_{21}\\
 x_{23}   &  9.96103440966 & -13.04 x_{5} & +  1.90 x_{26} & +  1.22 x_{28} & -11.38 x_{7} & +  1.66 x_{32} & -0.17 x_{30} & +  0.61 x_{27} & -41.22 x_{4} & -27.86 x_{8} & -31.00 x_{10} & -5.31 x_{17} & -20.06 x_{12} & -13.88 x_{13} & -43.51 x_{3} & +  0.04 x_{25} & +  0.78 x_{24} & -0.28 x_{21}\\
 x_{16}   &  0.718784717663 & -0.57 x_{5} & +  0.07 x_{26} & +  0.02 x_{28} & -0.91 x_{7} & +  0.03 x_{32} & -0.09 x_{30} & +  0.05 x_{27} & -1.14 x_{4} & +  0.36 x_{8} & -0.50 x_{10} & -0.21 x_{17} & -1.11 x_{12} & -0.62 x_{13} & -1.82 x_{3} & +  0.06 x_{25} & +  0.08 x_{24} & -0.07 x_{21}\\
 x_{15}   &  0.617313987394 & -0.24 x_{5} & +  0.10 x_{26} & +  0.03 x_{28} & -0.90 x_{7} & +  0.06 x_{32} & -0.07 x_{30} & +  0.02 x_{27} & -1.50 x_{4} & -0.00 x_{8} & -0.51 x_{10} & +  0.11 x_{17} & -0.34 x_{12} & -0.36 x_{13} & -1.98 x_{3} & +  0.07 x_{25} & +  0.09 x_{24} & -0.01 x_{21}\\
 x_{14}   &  0.683156759279 & -0.33 x_{5} & +  0.09 x_{26} & +  0.06 x_{28} & -1.14 x_{7} & +  0.08 x_{32} & -0.06 x_{30} & +  0.01 x_{27} & -1.99 x_{4} & -0.96 x_{8} & -0.99 x_{10} & -0.17 x_{17} & -0.62 x_{12} & -0.28 x_{13} & -1.96 x_{3} & +  0.02 x_{25} & +  0.02 x_{24} & -0.02 x_{21}\\
 x_{6}   &  0.615279016984 & -0.29 x_{5} & +  0.03 x_{26} & -0.02 x_{28} & -0.52 x_{7} & +  0.01 x_{32} & -0.10 x_{30} & -0.00 x_{27} & -0.31 x_{4} & +  0.81 x_{8} & +  1.19 x_{10} & -0.21 x_{17} & +  0.14 x_{12} & -0.51 x_{13} & -1.41 x_{3} & +  0.11 x_{25} & +  0.03 x_{24} & -0.00 x_{21}\\
 x_{11}   &  0.0144571417913 & +  0.44 x_{5} & -0.01 x_{26} & +  0.01 x_{28} & -0.25 x_{7} & -0.04 x_{32} & -0.03 x_{30} & +  0.01 x_{27} & +  0.28 x_{4} & +  0.23 x_{8} & +  1.00 x_{10} & +  0.24 x_{17} & +  0.49 x_{12} & -0.73 x_{13} & +  0.17 x_{3} & +  0.01 x_{25} & -0.02 x_{24} & +  0.04 x_{21}\\
 x_{29}   &  7.70419752447 & -7.26 x_{5} & +  1.26 x_{26} & -0.20 x_{28} & -11.36 x_{7} & +  0.32 x_{32} & -0.87 x_{30} & +  0.06 x_{27} & -13.87 x_{4} & +  6.94 x_{8} & -2.91 x_{10} & +  5.25 x_{17} & -10.11 x_{12} & +  3.40 x_{13} & -32.58 x_{3} & +  0.91 x_{25} & +  0.38 x_{24} & +  0.06 x_{21}\\
 x_{20}   &  6.03507107994 & -2.85 x_{5} & +  0.79 x_{26} & +  0.01 x_{28} & +  4.21 x_{7} & +  0.26 x_{32} & -1.40 x_{30} & -0.18 x_{27} & -2.09 x_{4} & + 15.08 x_{8} & + 10.78 x_{10} & -10.09 x_{17} & +  2.04 x_{12} & +  3.62 x_{13} & -27.03 x_{3} & +  1.42 x_{25} & +  1.12 x_{24} & -0.43 x_{21}\\
 x_{31}   &  6.76299663943 & -7.13 x_{5} & +  0.86 x_{26} & +  0.16 x_{28} & -5.02 x_{7} & +  0.73 x_{32} & -0.35 x_{30} & -0.31 x_{27} & -10.23 x_{4} & -1.17 x_{8} & +  4.55 x_{10} & -1.90 x_{17} & + 12.50 x_{12} & +  5.66 x_{13} & -19.33 x_{3} & +  0.49 x_{25} & +  0.28 x_{24} & +  0.52 x_{21}\\
 x_{9}   &  0.366659044217 & -0.23 x_{5} & +  0.09 x_{26} & +  0.02 x_{28} & -1.13 x_{7} & +  0.09 x_{32} & -0.04 x_{30} & +  0.07 x_{27} & -1.67 x_{4} & -0.69 x_{8} & -1.17 x_{10} & +  0.73 x_{17} & -1.34 x_{12} & -0.40 x_{13} & -2.21 x_{3} & +  0.02 x_{25} & +  0.08 x_{24} & -0.01 x_{21}\\
\hline
z    &  9.16507552214 & -6.86 x_{5} & +  1.27 x_{26} & +  0.62 x_{28} & -19.41 x_{7} & +  0.74 x_{32} & -0.83 x_{30} & +  0.45 x_{27} & -18.82 x_{4} & -2.90 x_{8} & -9.05 x_{10} & -2.16 x_{17} & -9.77 x_{12} & -6.75 x_{13} & -26.24 x_{3} & +  0.47 x_{25} & +  0.55 x_{24} & -0.37 x_{21}\\
\end{array}\]


 $ x_{24} $ enters and $ x_{11} $ leaves 

 \[\begin{array}{c| c c@{\hskip 2pt} c@{\hskip 2pt} c@{\hskip 2pt} c@{\hskip 2pt} c@{\hskip 2pt} c@{\hskip 2pt} c@{\hskip 2pt} c@{\hskip 2pt} c@{\hskip 2pt} c@{\hskip 2pt} c@{\hskip 2pt} c@{\hskip 2pt} c@{\hskip 2pt} c@{\hskip 2pt} c@{\hskip 2pt} c@{\hskip 2pt} c@{\hskip 2pt} }
 x_{18}   &  7.45705903766 & -7.87 x_{5} & +  0.76 x_{26} & +  0.57 x_{28} & -9.01 x_{7} & +  0.66 x_{32} & +  0.18 x_{30} & +  0.37 x_{27} & -16.55 x_{4} & -15.85 x_{8} & -21.34 x_{10} & -11.01 x_{17} & -15.83 x_{12} & -4.86 x_{13} & -4.83 x_{3} & +  0.00 x_{25} & +  6.77 x_{11} & -0.30 x_{21}\\
 x_{1}   &  1.36143195645 & +  0.25 x_{5} & +  0.15 x_{26} & +  0.07 x_{28} & -3.52 x_{7} & -0.08 x_{32} & -0.26 x_{30} & +  0.12 x_{27} & -1.48 x_{4} & -0.12 x_{8} & +  2.63 x_{10} & +  1.26 x_{17} & -0.29 x_{12} & -3.93 x_{13} & -3.50 x_{3} & +  0.11 x_{25} & -4.32 x_{11} & +  0.08 x_{21}\\
 x_{19}   &  14.7250154139 & + 12.92 x_{5} & +  1.27 x_{26} & +  1.09 x_{28} & -26.88 x_{7} & -1.69 x_{32} & -3.08 x_{30} & +  1.33 x_{27} & -0.14 x_{4} & +  6.70 x_{8} & + 49.25 x_{10} & +  2.30 x_{17} & + 16.47 x_{12} & -47.33 x_{13} & -34.61 x_{3} & +  1.73 x_{25} & -52.10 x_{11} & +  1.00 x_{21}\\
 x_{2}   &  0.104927752711 & +  0.10 x_{5} & -0.01 x_{26} & +  0.05 x_{28} & -0.02 x_{7} & +  0.02 x_{32} & +  0.03 x_{30} & +  0.04 x_{27} & -0.44 x_{4} & -0.44 x_{8} & -0.14 x_{10} & -0.56 x_{17} & -0.12 x_{12} & -0.38 x_{13} & +  0.12 x_{3} & +  0.00 x_{25} & -0.19 x_{11} & -0.03 x_{21}\\
 x_{22}   &  13.869522417 & + 19.43 x_{5} & +  1.57 x_{26} & +  0.89 x_{28} & -39.72 x_{7} & -0.76 x_{32} & -2.77 x_{30} & +  1.42 x_{27} & -18.60 x_{4} & -1.48 x_{8} & + 36.33 x_{10} & + 24.62 x_{17} & -5.71 x_{12} & -39.83 x_{13} & -43.54 x_{3} & +  1.14 x_{25} & -47.12 x_{11} & +  1.54 x_{21}\\
 x_{23}   &  10.5785526229 & +  5.82 x_{5} & +  1.49 x_{26} & +  1.54 x_{28} & -21.91 x_{7} & -0.19 x_{32} & -1.63 x_{30} & +  0.98 x_{27} & -29.39 x_{4} & -18.06 x_{8} & + 11.65 x_{10} & +  4.76 x_{17} & +  0.85 x_{12} & -45.27 x_{13} & -36.39 x_{3} & +  0.34 x_{25} & -42.71 x_{11} & +  1.22 x_{21}\\
 x_{16}   &  0.781782225165 & +  1.35 x_{5} & +  0.03 x_{26} & +  0.05 x_{28} & -1.98 x_{7} & -0.15 x_{32} & -0.23 x_{30} & +  0.08 x_{27} & +  0.07 x_{4} & +  1.36 x_{8} & +  3.85 x_{10} & +  0.82 x_{17} & +  1.02 x_{12} & -3.83 x_{13} & -1.09 x_{3} & +  0.09 x_{25} & -4.36 x_{11} & +  0.08 x_{21}\\
 x_{15}   &  0.684969716182 & +  1.83 x_{5} & +  0.06 x_{26} & +  0.06 x_{28} & -2.06 x_{7} & -0.15 x_{32} & -0.22 x_{30} & +  0.06 x_{27} & -0.20 x_{4} & +  1.07 x_{8} & +  4.17 x_{10} & +  1.21 x_{17} & +  1.95 x_{12} & -3.80 x_{13} & -1.20 x_{3} & +  0.11 x_{25} & -4.68 x_{11} & +  0.15 x_{21}\\
 x_{14}   &  0.699962590183 & +  0.18 x_{5} & +  0.08 x_{26} & +  0.07 x_{28} & -1.43 x_{7} & +  0.03 x_{32} & -0.10 x_{30} & +  0.02 x_{27} & -1.67 x_{4} & -0.69 x_{8} & +  0.17 x_{10} & +  0.11 x_{17} & -0.05 x_{12} & -1.13 x_{13} & -1.76 x_{3} & +  0.03 x_{25} & -1.16 x_{11} & +  0.02 x_{21}\\
 x_{6}   &  0.640553316041 & +  0.48 x_{5} & +  0.02 x_{26} & -0.00 x_{28} & -0.95 x_{7} & -0.07 x_{32} & -0.16 x_{30} & +  0.01 x_{27} & +  0.17 x_{4} & +  1.21 x_{8} & +  2.93 x_{10} & +  0.21 x_{17} & +  1.00 x_{12} & -1.80 x_{13} & -1.11 x_{3} & +  0.12 x_{25} & -1.75 x_{11} & +  0.06 x_{21}\\
 x_{24}   &  0.79441525837 & + 24.26 x_{5} & -0.52 x_{26} & +  0.40 x_{28} & -13.55 x_{7} & -2.39 x_{32} & -1.87 x_{30} & +  0.48 x_{27} & + 15.22 x_{4} & + 12.60 x_{8} & + 54.87 x_{10} & + 12.95 x_{17} & + 26.89 x_{12} & -40.38 x_{13} & +  9.16 x_{3} & +  0.38 x_{25} & -54.95 x_{11} & +  1.93 x_{21}\\
 x_{29}   &  8.00545142298 & +  1.94 x_{5} & +  1.06 x_{26} & -0.04 x_{28} & -16.50 x_{7} & -0.58 x_{32} & -1.58 x_{30} & +  0.24 x_{27} & -8.10 x_{4} & + 11.72 x_{8} & + 17.89 x_{10} & + 10.16 x_{17} & +  0.09 x_{12} & -11.91 x_{13} & -29.10 x_{3} & +  1.05 x_{25} & -20.84 x_{11} & +  0.80 x_{21}\\
 x_{20}   &  6.92136389342 & + 24.21 x_{5} & +  0.20 x_{26} & +  0.46 x_{28} & -10.90 x_{7} & -2.40 x_{32} & -3.48 x_{30} & +  0.35 x_{27} & + 14.90 x_{4} & + 29.14 x_{8} & + 71.99 x_{10} & +  4.36 x_{17} & + 32.04 x_{12} & -41.43 x_{13} & -16.81 x_{3} & +  1.84 x_{25} & -61.30 x_{11} & +  1.73 x_{21}\\
 x_{31}   &  6.98532017296 & -0.34 x_{5} & +  0.72 x_{26} & +  0.27 x_{28} & -8.82 x_{7} & +  0.06 x_{32} & -0.87 x_{30} & -0.18 x_{27} & -5.97 x_{4} & +  2.35 x_{8} & + 19.90 x_{10} & +  1.73 x_{17} & + 20.02 x_{12} & -5.64 x_{13} & -16.77 x_{3} & +  0.60 x_{25} & -15.38 x_{11} & +  1.06 x_{21}\\
 x_{9}   &  0.426797585354 & +  1.61 x_{5} & +  0.05 x_{26} & +  0.05 x_{28} & -2.15 x_{7} & -0.09 x_{32} & -0.18 x_{30} & +  0.10 x_{27} & -0.52 x_{4} & +  0.27 x_{8} & +  2.98 x_{10} & +  1.71 x_{17} & +  0.69 x_{12} & -3.46 x_{13} & -1.51 x_{3} & +  0.04 x_{25} & -4.16 x_{11} & +  0.13 x_{21}\\
\hline
z    &  9.59960595441 & +  6.41 x_{5} & +  0.98 x_{26} & +  0.84 x_{28} & -26.82 x_{7} & -0.57 x_{32} & -1.85 x_{30} & +  0.71 x_{27} & -10.50 x_{4} & +  4.00 x_{8} & + 20.96 x_{10} & +  4.92 x_{17} & +  4.94 x_{12} & -28.83 x_{13} & -21.23 x_{3} & +  0.68 x_{25} & -30.06 x_{11} & +  0.69 x_{21}\\
\end{array}\]


 $ x_{5} $ enters and $ x_{18} $ leaves 

 \[\begin{array}{c| c c@{\hskip 2pt} c@{\hskip 2pt} c@{\hskip 2pt} c@{\hskip 2pt} c@{\hskip 2pt} c@{\hskip 2pt} c@{\hskip 2pt} c@{\hskip 2pt} c@{\hskip 2pt} c@{\hskip 2pt} c@{\hskip 2pt} c@{\hskip 2pt} c@{\hskip 2pt} c@{\hskip 2pt} c@{\hskip 2pt} c@{\hskip 2pt} c@{\hskip 2pt} }
 x_{5}   &  0.94777008903 & -0.13 x_{18} & +  0.10 x_{26} & +  0.07 x_{28} & -1.14 x_{7} & +  0.08 x_{32} & +  0.02 x_{30} & +  0.05 x_{27} & -2.10 x_{4} & -2.02 x_{8} & -2.71 x_{10} & -1.40 x_{17} & -2.01 x_{12} & -0.62 x_{13} & -0.61 x_{3} & +  0.00 x_{25} & +  0.86 x_{11} & -0.04 x_{21}\\
 x_{1}   &  1.60196679211 & -0.03 x_{18} & +  0.18 x_{26} & +  0.08 x_{28} & -3.81 x_{7} & -0.06 x_{32} & -0.25 x_{30} & +  0.14 x_{27} & -2.01 x_{4} & -0.63 x_{8} & +  1.95 x_{10} & +  0.91 x_{17} & -0.80 x_{12} & -4.09 x_{13} & -3.65 x_{3} & +  0.11 x_{25} & -4.10 x_{11} & +  0.07 x_{21}\\
 x_{19}   &  26.96736564 & -1.64 x_{18} & +  2.53 x_{26} & +  2.02 x_{28} & -41.67 x_{7} & -0.60 x_{32} & -2.79 x_{30} & +  1.95 x_{27} & -27.30 x_{4} & -19.33 x_{8} & + 14.22 x_{10} & -15.77 x_{17} & -9.53 x_{12} & -55.31 x_{13} & -42.54 x_{3} & +  1.73 x_{25} & -40.99 x_{11} & +  0.52 x_{21}\\
 x_{2}   &  0.201557253103 & -0.01 x_{18} & -0.00 x_{26} & +  0.06 x_{28} & -0.14 x_{7} & +  0.02 x_{32} & +  0.03 x_{30} & +  0.05 x_{27} & -0.65 x_{4} & -0.65 x_{8} & -0.41 x_{10} & -0.70 x_{17} & -0.32 x_{12} & -0.45 x_{13} & +  0.06 x_{3} & +  0.00 x_{25} & -0.10 x_{11} & -0.03 x_{21}\\
 x_{22}   &  32.2811292103 & -2.47 x_{18} & +  3.46 x_{26} & +  2.29 x_{28} & -61.96 x_{7} & +  0.87 x_{32} & -2.34 x_{30} & +  2.35 x_{27} & -59.45 x_{4} & -40.62 x_{8} & -16.35 x_{10} & -2.56 x_{17} & -44.80 x_{12} & -51.84 x_{13} & -55.46 x_{3} & +  1.14 x_{25} & -30.40 x_{11} & +  0.80 x_{21}\\
 x_{23}   &  16.0955283622 & -0.74 x_{18} & +  2.06 x_{26} & +  1.96 x_{28} & -28.57 x_{7} & +  0.30 x_{32} & -1.50 x_{30} & +  1.25 x_{27} & -41.63 x_{4} & -29.79 x_{8} & -4.14 x_{10} & -3.39 x_{17} & -10.86 x_{12} & -48.87 x_{13} & -39.96 x_{3} & +  0.34 x_{25} & -37.70 x_{11} & +  1.00 x_{21}\\
 x_{16}   &  2.06585666949 & -0.17 x_{18} & +  0.16 x_{26} & +  0.15 x_{28} & -3.53 x_{7} & -0.04 x_{32} & -0.20 x_{30} & +  0.15 x_{27} & -2.78 x_{4} & -1.37 x_{8} & +  0.18 x_{10} & -1.08 x_{17} & -1.71 x_{12} & -4.66 x_{13} & -1.93 x_{3} & +  0.09 x_{25} & -3.19 x_{11} & +  0.03 x_{21}\\
 x_{15}   &  2.41950380022 & -0.23 x_{18} & +  0.24 x_{26} & +  0.19 x_{28} & -4.15 x_{7} & +  0.01 x_{32} & -0.18 x_{30} & +  0.14 x_{27} & -4.05 x_{4} & -2.62 x_{8} & -0.80 x_{10} & -1.35 x_{17} & -1.74 x_{12} & -4.93 x_{13} & -2.32 x_{3} & +  0.11 x_{25} & -3.10 x_{11} & +  0.08 x_{21}\\
 x_{14}   &  0.871360956486 & -0.02 x_{18} & +  0.09 x_{26} & +  0.08 x_{28} & -1.64 x_{7} & +  0.05 x_{32} & -0.09 x_{30} & +  0.03 x_{27} & -2.05 x_{4} & -1.05 x_{8} & -0.32 x_{10} & -0.15 x_{17} & -0.41 x_{12} & -1.24 x_{13} & -1.87 x_{3} & +  0.03 x_{25} & -1.01 x_{11} & +  0.01 x_{21}\\
 x_{6}   &  1.09278810627 & -0.06 x_{18} & +  0.06 x_{26} & +  0.03 x_{28} & -1.50 x_{7} & -0.03 x_{32} & -0.15 x_{30} & +  0.04 x_{27} & -0.83 x_{4} & +  0.25 x_{8} & +  1.64 x_{10} & -0.46 x_{17} & +  0.04 x_{12} & -2.09 x_{13} & -1.41 x_{3} & +  0.12 x_{25} & -1.34 x_{11} & +  0.04 x_{21}\\
 x_{24}   &  23.7864763266 & -3.08 x_{18} & +  1.84 x_{26} & +  2.15 x_{28} & -41.32 x_{7} & -0.35 x_{32} & -1.33 x_{30} & +  1.63 x_{27} & -35.79 x_{4} & -36.28 x_{8} & -10.92 x_{10} & -21.00 x_{17} & -21.93 x_{12} & -55.37 x_{13} & -5.73 x_{3} & +  0.38 x_{25} & -34.07 x_{11} & +  1.02 x_{21}\\
 x_{29}   &  9.84237178609 & -0.25 x_{18} & +  1.25 x_{26} & +  0.10 x_{28} & -18.72 x_{7} & -0.42 x_{32} & -1.53 x_{30} & +  0.33 x_{27} & -12.18 x_{4} & +  7.81 x_{8} & + 12.64 x_{10} & +  7.45 x_{17} & -3.81 x_{12} & -13.11 x_{13} & -30.29 x_{3} & +  1.05 x_{25} & -19.17 x_{11} & +  0.72 x_{21}\\
 x_{20}   &  29.8704821949 & -3.08 x_{18} & +  2.56 x_{26} & +  2.21 x_{28} & -38.63 x_{7} & -0.37 x_{32} & -2.94 x_{30} & +  1.50 x_{27} & -36.02 x_{4} & -19.66 x_{8} & +  6.33 x_{10} & -29.52 x_{17} & -16.68 x_{12} & -56.39 x_{13} & -31.67 x_{3} & +  1.85 x_{25} & -40.46 x_{11} & +  0.81 x_{21}\\
 x_{31}   &  6.66378348872 & +  0.04 x_{18} & +  0.68 x_{26} & +  0.25 x_{28} & -8.43 x_{7} & +  0.03 x_{32} & -0.88 x_{30} & -0.19 x_{27} & -5.26 x_{4} & +  3.04 x_{8} & + 20.82 x_{10} & +  2.20 x_{17} & + 20.71 x_{12} & -5.43 x_{13} & -16.56 x_{3} & +  0.60 x_{25} & -15.67 x_{11} & +  1.07 x_{21}\\
 x_{9}   &  1.9491520098 & -0.20 x_{18} & +  0.21 x_{26} & +  0.16 x_{28} & -3.99 x_{7} & +  0.04 x_{32} & -0.14 x_{30} & +  0.18 x_{27} & -3.90 x_{4} & -2.97 x_{8} & -1.37 x_{10} & -0.54 x_{17} & -2.54 x_{12} & -4.45 x_{13} & -2.50 x_{3} & +  0.04 x_{25} & -2.78 x_{11} & +  0.07 x_{21}\\
\hline
z    &  15.6719888402 & -0.81 x_{18} & +  1.61 x_{26} & +  1.30 x_{28} & -34.16 x_{7} & -0.03 x_{32} & -1.71 x_{30} & +  1.01 x_{27} & -23.97 x_{4} & -8.91 x_{8} & +  3.59 x_{10} & -4.04 x_{17} & -7.95 x_{12} & -32.79 x_{13} & -25.16 x_{3} & +  0.68 x_{25} & -24.54 x_{11} & +  0.44 x_{21}\\
\end{array}\]


 $ x_{10} $ enters and $ x_{5} $ leaves 

 \[\begin{array}{c| c c@{\hskip 2pt} c@{\hskip 2pt} c@{\hskip 2pt} c@{\hskip 2pt} c@{\hskip 2pt} c@{\hskip 2pt} c@{\hskip 2pt} c@{\hskip 2pt} c@{\hskip 2pt} c@{\hskip 2pt} c@{\hskip 2pt} c@{\hskip 2pt} c@{\hskip 2pt} c@{\hskip 2pt} c@{\hskip 2pt} c@{\hskip 2pt} c@{\hskip 2pt} }
 x_{10}   &  0.34949047175 & -0.05 x_{18} & +  0.04 x_{26} & +  0.03 x_{28} & -0.42 x_{7} & +  0.03 x_{32} & +  0.01 x_{30} & +  0.02 x_{27} & -0.78 x_{4} & -0.74 x_{8} & -0.37 x_{5} & -0.52 x_{17} & -0.74 x_{12} & -0.23 x_{13} & -0.23 x_{3} & +  0.00 x_{25} & +  0.32 x_{11} & -0.01 x_{21}\\
 x_{1}   &  2.2822367051 & -0.12 x_{18} & +  0.25 x_{26} & +  0.14 x_{28} & -4.63 x_{7} & +  0.00 x_{32} & -0.23 x_{30} & +  0.17 x_{27} & -3.52 x_{4} & -2.08 x_{8} & -0.72 x_{5} & -0.10 x_{17} & -2.25 x_{12} & -4.53 x_{13} & -4.10 x_{3} & +  0.11 x_{25} & -3.48 x_{11} & +  0.05 x_{21}\\
 x_{19}   &  31.9383133739 & -2.31 x_{18} & +  3.04 x_{26} & +  2.40 x_{28} & -47.68 x_{7} & -0.16 x_{32} & -2.67 x_{30} & +  2.20 x_{27} & -38.33 x_{4} & -29.89 x_{8} & -5.24 x_{5} & -23.11 x_{17} & -20.08 x_{12} & -58.55 x_{13} & -45.76 x_{3} & +  1.73 x_{25} & -36.47 x_{11} & +  0.32 x_{21}\\
 x_{2}   &  0.0569718858549 & +  0.01 x_{18} & -0.02 x_{26} & +  0.05 x_{28} & +  0.04 x_{7} & +  0.01 x_{32} & +  0.02 x_{30} & +  0.04 x_{27} & -0.33 x_{4} & -0.34 x_{8} & +  0.15 x_{5} & -0.49 x_{17} & -0.02 x_{12} & -0.35 x_{13} & +  0.15 x_{3} & +  0.00 x_{25} & -0.23 x_{11} & -0.03 x_{21}\\
 x_{22}   &  26.5660644454 & -1.70 x_{18} & +  2.88 x_{26} & +  1.86 x_{28} & -55.06 x_{7} & +  0.36 x_{32} & -2.47 x_{30} & +  2.06 x_{27} & -46.77 x_{4} & -28.47 x_{8} & +  6.03 x_{5} & +  5.88 x_{17} & -32.67 x_{12} & -48.11 x_{13} & -51.76 x_{3} & +  1.14 x_{25} & -35.59 x_{11} & +  1.03 x_{21}\\
 x_{23}   &  14.6492999985 & -0.55 x_{18} & +  1.91 x_{26} & +  1.85 x_{28} & -26.82 x_{7} & +  0.17 x_{32} & -1.53 x_{30} & +  1.18 x_{27} & -38.42 x_{4} & -26.71 x_{8} & +  1.53 x_{5} & -1.25 x_{17} & -7.79 x_{12} & -47.93 x_{13} & -39.02 x_{3} & +  0.34 x_{25} & -39.02 x_{11} & +  1.06 x_{21}\\
 x_{16}   &  2.1271289091 & -0.18 x_{18} & +  0.17 x_{26} & +  0.16 x_{28} & -3.61 x_{7} & -0.04 x_{32} & -0.20 x_{30} & +  0.15 x_{27} & -2.92 x_{4} & -1.50 x_{8} & -0.06 x_{5} & -1.17 x_{17} & -1.84 x_{12} & -4.70 x_{13} & -1.97 x_{3} & +  0.09 x_{25} & -3.14 x_{11} & +  0.02 x_{21}\\
 x_{15}   &  2.14100318638 & -0.20 x_{18} & +  0.21 x_{26} & +  0.17 x_{28} & -3.82 x_{7} & -0.02 x_{32} & -0.19 x_{30} & +  0.13 x_{27} & -3.43 x_{4} & -2.02 x_{8} & +  0.29 x_{5} & -0.93 x_{17} & -1.15 x_{12} & -4.74 x_{13} & -2.14 x_{3} & +  0.11 x_{25} & -3.36 x_{11} & +  0.09 x_{21}\\
 x_{14}   &  0.759053153041 & -0.01 x_{18} & +  0.08 x_{26} & +  0.07 x_{28} & -1.50 x_{7} & +  0.04 x_{32} & -0.09 x_{30} & +  0.02 x_{27} & -1.80 x_{4} & -0.82 x_{8} & +  0.12 x_{5} & +  0.02 x_{17} & -0.18 x_{12} & -1.17 x_{13} & -1.80 x_{3} & +  0.03 x_{25} & -1.11 x_{11} & +  0.02 x_{21}\\
 x_{6}   &  1.66583515683 & -0.14 x_{18} & +  0.12 x_{26} & +  0.07 x_{28} & -2.19 x_{7} & +  0.02 x_{32} & -0.14 x_{30} & +  0.07 x_{27} & -2.10 x_{4} & -0.97 x_{8} & -0.60 x_{5} & -1.31 x_{17} & -1.18 x_{12} & -2.47 x_{13} & -1.78 x_{3} & +  0.12 x_{25} & -0.82 x_{11} & +  0.02 x_{21}\\
 x_{24}   &  19.9707341743 & -2.57 x_{18} & +  1.44 x_{26} & +  1.86 x_{28} & -36.71 x_{7} & -0.69 x_{32} & -1.42 x_{30} & +  1.44 x_{27} & -27.33 x_{4} & -28.17 x_{8} & +  4.03 x_{5} & -15.36 x_{17} & -13.82 x_{12} & -52.88 x_{13} & -3.26 x_{3} & +  0.38 x_{25} & -37.53 x_{11} & +  1.17 x_{21}\\
 x_{29}   &  14.2595629161 & -0.84 x_{18} & +  1.70 x_{26} & +  0.43 x_{28} & -24.05 x_{7} & -0.03 x_{32} & -1.43 x_{30} & +  0.55 x_{27} & -21.98 x_{4} & -1.58 x_{8} & -4.66 x_{5} & +  0.92 x_{17} & -13.19 x_{12} & -15.99 x_{13} & -33.16 x_{3} & +  1.05 x_{25} & -15.16 x_{11} & +  0.55 x_{21}\\
 x_{20}   &  32.0814234903 & -3.37 x_{18} & +  2.78 x_{26} & +  2.38 x_{28} & -41.30 x_{7} & -0.17 x_{32} & -2.89 x_{30} & +  1.61 x_{27} & -40.93 x_{4} & -24.36 x_{8} & -2.33 x_{5} & -32.79 x_{17} & -21.38 x_{12} & -57.83 x_{13} & -33.11 x_{3} & +  1.85 x_{25} & -38.45 x_{11} & +  0.73 x_{21}\\
 x_{31}   &  13.9412543026 & -0.93 x_{18} & +  1.43 x_{26} & +  0.80 x_{28} & -17.22 x_{7} & +  0.67 x_{32} & -0.71 x_{30} & +  0.17 x_{27} & -21.40 x_{4} & -12.44 x_{8} & -7.68 x_{5} & -8.54 x_{17} & +  5.25 x_{12} & -10.17 x_{13} & -21.27 x_{3} & +  0.60 x_{25} & -9.06 x_{11} & +  0.78 x_{21}\\
 x_{9}   &  1.46876347821 & -0.14 x_{18} & +  0.16 x_{26} & +  0.13 x_{28} & -3.41 x_{7} & -0.00 x_{32} & -0.15 x_{30} & +  0.15 x_{27} & -2.83 x_{4} & -1.95 x_{8} & +  0.51 x_{5} & +  0.17 x_{17} & -1.52 x_{12} & -4.14 x_{13} & -2.19 x_{3} & +  0.04 x_{25} & -3.21 x_{11} & +  0.09 x_{21}\\
\hline
z    &  16.9252718236 & -0.98 x_{18} & +  1.73 x_{26} & +  1.40 x_{28} & -35.67 x_{7} & +  0.08 x_{32} & -1.68 x_{30} & +  1.08 x_{27} & -26.75 x_{4} & -11.58 x_{8} & -1.32 x_{5} & -5.89 x_{17} & -10.62 x_{12} & -33.61 x_{13} & -25.97 x_{3} & +  0.68 x_{25} & -23.40 x_{11} & +  0.39 x_{21}\\
\end{array}\]


 $ x_{21} $ enters and $ x_{2} $ leaves 

 \[\begin{array}{c| c c@{\hskip 2pt} c@{\hskip 2pt} c@{\hskip 2pt} c@{\hskip 2pt} c@{\hskip 2pt} c@{\hskip 2pt} c@{\hskip 2pt} c@{\hskip 2pt} c@{\hskip 2pt} c@{\hskip 2pt} c@{\hskip 2pt} c@{\hskip 2pt} c@{\hskip 2pt} c@{\hskip 2pt} c@{\hskip 2pt} c@{\hskip 2pt} c@{\hskip 2pt} }
 x_{10}   &  0.318998586715 & -0.05 x_{18} & +  0.04 x_{26} & +  0.00 x_{28} & -0.44 x_{7} & +  0.02 x_{32} & -0.01 x_{30} & -0.00 x_{27} & -0.60 x_{4} & -0.56 x_{8} & -0.45 x_{5} & -0.25 x_{17} & -0.73 x_{12} & -0.04 x_{13} & -0.31 x_{3} & -0.00 x_{25} & +  0.44 x_{11} & +  0.54 x_{2}\\
 x_{1}   &  2.38370684434 & -0.11 x_{18} & +  0.22 x_{26} & +  0.22 x_{28} & -4.57 x_{7} & +  0.02 x_{32} & -0.19 x_{30} & +  0.24 x_{27} & -4.11 x_{4} & -2.68 x_{8} & -0.45 x_{5} & -0.97 x_{17} & -2.28 x_{12} & -5.16 x_{13} & -3.82 x_{3} & +  0.12 x_{25} & -3.90 x_{11} & -1.78 x_{2}\\
 x_{19}   &  32.6371895821 & -2.23 x_{18} & +  2.83 x_{26} & +  3.00 x_{28} & -47.21 x_{7} & -0.02 x_{32} & -2.37 x_{30} & +  2.68 x_{27} & -42.41 x_{4} & -34.07 x_{8} & -3.37 x_{5} & -29.13 x_{17} & -20.29 x_{12} & -62.88 x_{13} & -43.88 x_{3} & +  1.78 x_{25} & -39.34 x_{11} & -12.27 x_{2}\\
 x_{21}   &  2.19604280234 & +  0.25 x_{18} & -0.65 x_{26} & +  1.91 x_{28} & +  1.48 x_{7} & +  0.45 x_{32} & +  0.95 x_{30} & +  1.50 x_{27} & -12.82 x_{4} & -13.12 x_{8} & +  5.88 x_{5} & -18.92 x_{17} & -0.64 x_{12} & -13.60 x_{13} & +  5.90 x_{3} & +  0.16 x_{25} & -9.02 x_{11} & -38.55 x_{2}\\
 x_{22}   &  28.8321219463 & -1.45 x_{18} & +  2.21 x_{26} & +  3.83 x_{28} & -53.53 x_{7} & +  0.83 x_{32} & -1.49 x_{30} & +  3.61 x_{27} & -60.00 x_{4} & -42.01 x_{8} & + 12.10 x_{5} & -13.64 x_{17} & -33.33 x_{12} & -62.14 x_{13} & -45.67 x_{3} & +  1.30 x_{25} & -44.90 x_{11} & -39.78 x_{2}\\
 x_{23}   &  16.977286493 & -0.28 x_{18} & +  1.22 x_{26} & +  3.87 x_{28} & -25.26 x_{7} & +  0.65 x_{32} & -0.52 x_{30} & +  2.77 x_{27} & -52.02 x_{4} & -40.63 x_{8} & +  7.76 x_{5} & -21.31 x_{17} & -8.47 x_{12} & -62.34 x_{13} & -32.77 x_{3} & +  0.51 x_{25} & -48.58 x_{11} & -40.86 x_{2}\\
 x_{16}   &  2.18180900464 & -0.17 x_{18} & +  0.15 x_{26} & +  0.20 x_{28} & -3.57 x_{7} & -0.02 x_{32} & -0.18 x_{30} & +  0.19 x_{27} & -3.23 x_{4} & -1.82 x_{8} & +  0.08 x_{5} & -1.64 x_{17} & -1.85 x_{12} & -5.04 x_{13} & -1.82 x_{3} & +  0.09 x_{25} & -3.36 x_{11} & -0.96 x_{2}\\
 x_{15}   &  2.34847567131 & -0.17 x_{18} & +  0.15 x_{26} & +  0.35 x_{28} & -3.68 x_{7} & +  0.02 x_{32} & -0.10 x_{30} & +  0.27 x_{27} & -4.64 x_{4} & -3.26 x_{8} & +  0.85 x_{5} & -2.72 x_{17} & -1.21 x_{12} & -6.03 x_{13} & -1.58 x_{3} & +  0.12 x_{25} & -4.21 x_{11} & -3.64 x_{2}\\
 x_{14}   &  0.799818291944 & -0.00 x_{18} & +  0.07 x_{26} & +  0.11 x_{28} & -1.47 x_{7} & +  0.05 x_{32} & -0.08 x_{30} & +  0.05 x_{27} & -2.04 x_{4} & -1.06 x_{8} & +  0.23 x_{5} & -0.33 x_{17} & -0.19 x_{12} & -1.42 x_{13} & -1.69 x_{3} & +  0.03 x_{25} & -1.28 x_{11} & -0.72 x_{2}\\
 x_{6}   &  1.70684433677 & -0.13 x_{18} & +  0.11 x_{26} & +  0.11 x_{28} & -2.16 x_{7} & +  0.03 x_{32} & -0.12 x_{30} & +  0.09 x_{27} & -2.34 x_{4} & -1.21 x_{8} & -0.49 x_{5} & -1.66 x_{17} & -1.19 x_{12} & -2.72 x_{13} & -1.67 x_{3} & +  0.12 x_{25} & -0.99 x_{11} & -0.72 x_{2}\\
 x_{24}   &  22.5437108823 & -2.28 x_{18} & +  0.69 x_{26} & +  4.10 x_{28} & -34.98 x_{7} & -0.16 x_{32} & -0.30 x_{30} & +  3.20 x_{27} & -42.35 x_{4} & -43.54 x_{8} & + 10.92 x_{5} & -37.53 x_{17} & -14.58 x_{12} & -68.81 x_{13} & +  3.66 x_{3} & +  0.57 x_{25} & -48.10 x_{11} & -45.16 x_{2}\\
 x_{29}   &  15.4630526953 & -0.70 x_{18} & +  1.35 x_{26} & +  1.48 x_{28} & -23.24 x_{7} & +  0.22 x_{32} & -0.91 x_{30} & +  1.38 x_{27} & -29.00 x_{4} & -8.77 x_{8} & -1.44 x_{5} & -9.44 x_{17} & -13.54 x_{12} & -23.44 x_{13} & -29.92 x_{3} & +  1.14 x_{25} & -20.10 x_{11} & -21.12 x_{2}\\
 x_{20}   &  33.6745406824 & -3.19 x_{18} & +  2.32 x_{26} & +  3.76 x_{28} & -40.23 x_{7} & +  0.16 x_{32} & -2.20 x_{30} & +  2.70 x_{27} & -50.23 x_{4} & -33.88 x_{8} & +  1.93 x_{5} & -46.51 x_{17} & -21.85 x_{12} & -67.69 x_{13} & -28.83 x_{3} & +  1.96 x_{25} & -45.00 x_{11} & -27.96 x_{2}\\
 x_{31}   &  15.6622249142 & -0.74 x_{18} & +  0.93 x_{26} & +  2.30 x_{28} & -16.06 x_{7} & +  1.03 x_{32} & +  0.04 x_{30} & +  1.35 x_{27} & -31.45 x_{4} & -22.72 x_{8} & -3.07 x_{5} & -23.37 x_{17} & +  4.75 x_{12} & -20.83 x_{13} & -16.65 x_{3} & +  0.72 x_{25} & -16.13 x_{11} & -30.21 x_{2}\\
 x_{9}   &  1.6724207551 & -0.12 x_{18} & +  0.10 x_{26} & +  0.31 x_{28} & -3.27 x_{7} & +  0.04 x_{32} & -0.07 x_{30} & +  0.29 x_{27} & -4.02 x_{4} & -3.17 x_{8} & +  1.05 x_{5} & -1.58 x_{17} & -1.58 x_{12} & -5.40 x_{13} & -1.64 x_{3} & +  0.06 x_{25} & -4.05 x_{11} & -3.57 x_{2}\\
\hline
z    &  17.792146174 & -0.88 x_{18} & +  1.48 x_{26} & +  2.15 x_{28} & -35.09 x_{7} & +  0.26 x_{32} & -1.30 x_{30} & +  1.67 x_{27} & -31.81 x_{4} & -16.76 x_{8} & +  1.00 x_{5} & -13.36 x_{17} & -10.87 x_{12} & -38.98 x_{13} & -23.65 x_{3} & +  0.74 x_{25} & -26.96 x_{11} & -15.22 x_{2}\\
\end{array}\]


 $ x_{5} $ enters and $ x_{10} $ leaves 

 \[\begin{array}{c| c c@{\hskip 2pt} c@{\hskip 2pt} c@{\hskip 2pt} c@{\hskip 2pt} c@{\hskip 2pt} c@{\hskip 2pt} c@{\hskip 2pt} c@{\hskip 2pt} c@{\hskip 2pt} c@{\hskip 2pt} c@{\hskip 2pt} c@{\hskip 2pt} c@{\hskip 2pt} c@{\hskip 2pt} c@{\hskip 2pt} c@{\hskip 2pt} c@{\hskip 2pt} }
 x_{5}   &  0.708259071444 & -0.11 x_{18} & +  0.10 x_{26} & +  0.00 x_{28} & -0.98 x_{7} & +  0.05 x_{32} & -0.01 x_{30} & -0.01 x_{27} & -1.33 x_{4} & -1.25 x_{8} & -2.22 x_{10} & -0.56 x_{17} & -1.63 x_{12} & -0.09 x_{13} & -0.68 x_{3} & -0.00 x_{25} & +  0.98 x_{11} & +  1.19 x_{2}\\
 x_{1}   &  2.06778610852 & -0.06 x_{18} & +  0.18 x_{26} & +  0.22 x_{28} & -4.13 x_{7} & -0.00 x_{32} & -0.18 x_{30} & +  0.24 x_{27} & -3.52 x_{4} & -2.13 x_{8} & +  0.99 x_{10} & -0.72 x_{17} & -1.55 x_{12} & -5.12 x_{13} & -3.52 x_{3} & +  0.12 x_{25} & -4.34 x_{11} & -2.31 x_{2}\\
 x_{19}   &  30.2478691238 & -1.85 x_{18} & +  2.50 x_{26} & +  3.00 x_{28} & -43.89 x_{7} & -0.20 x_{32} & -2.33 x_{30} & +  2.70 x_{27} & -37.93 x_{4} & -29.87 x_{8} & +  7.49 x_{10} & -27.23 x_{17} & -14.80 x_{12} & -62.59 x_{13} & -41.58 x_{3} & +  1.80 x_{25} & -42.66 x_{11} & -16.28 x_{2}\\
 x_{21}   &  6.3608384284 & -0.41 x_{18} & -0.06 x_{26} & +  1.91 x_{28} & -4.30 x_{7} & +  0.77 x_{32} & +  0.89 x_{30} & +  1.46 x_{27} & -20.62 x_{4} & -20.44 x_{8} & -13.06 x_{10} & -22.23 x_{17} & -10.21 x_{12} & -14.11 x_{13} & +  1.88 x_{3} & +  0.13 x_{25} & -3.24 x_{11} & -31.56 x_{2}\\
 x_{22}   &  37.4005109793 & -2.80 x_{18} & +  3.41 x_{26} & +  3.83 x_{28} & -65.42 x_{7} & +  1.49 x_{32} & -1.63 x_{30} & +  3.52 x_{27} & -76.05 x_{4} & -57.08 x_{8} & -26.86 x_{10} & -20.45 x_{17} & -53.02 x_{12} & -63.19 x_{13} & -53.95 x_{3} & +  1.25 x_{25} & -33.01 x_{11} & -25.40 x_{2}\\
 x_{23}   &  22.4730652937 & -1.15 x_{18} & +  1.99 x_{26} & +  3.87 x_{28} & -32.88 x_{7} & +  1.07 x_{32} & -0.60 x_{30} & +  2.72 x_{27} & -62.31 x_{4} & -50.29 x_{8} & -17.23 x_{10} & -25.67 x_{17} & -21.11 x_{12} & -63.01 x_{13} & -38.08 x_{3} & +  0.47 x_{25} & -40.95 x_{11} & -31.64 x_{2}\\
 x_{16}   &  2.23972168763 & -0.18 x_{18} & +  0.16 x_{26} & +  0.20 x_{28} & -3.65 x_{7} & -0.02 x_{32} & -0.18 x_{30} & +  0.19 x_{27} & -3.34 x_{4} & -1.93 x_{8} & -0.18 x_{10} & -1.68 x_{17} & -1.99 x_{12} & -5.05 x_{13} & -1.87 x_{3} & +  0.09 x_{25} & -3.28 x_{11} & -0.86 x_{2}\\
 x_{15}   &  2.95006799408 & -0.27 x_{18} & +  0.23 x_{26} & +  0.35 x_{28} & -4.51 x_{7} & +  0.07 x_{32} & -0.11 x_{30} & +  0.26 x_{27} & -5.77 x_{4} & -4.32 x_{8} & -1.89 x_{10} & -3.20 x_{17} & -2.59 x_{12} & -6.10 x_{13} & -2.17 x_{3} & +  0.12 x_{25} & -3.37 x_{11} & -2.63 x_{2}\\
 x_{14}   &  0.961055880141 & -0.03 x_{18} & +  0.09 x_{26} & +  0.11 x_{28} & -1.70 x_{7} & +  0.06 x_{32} & -0.08 x_{30} & +  0.05 x_{27} & -2.34 x_{4} & -1.34 x_{8} & -0.51 x_{10} & -0.46 x_{17} & -0.56 x_{12} & -1.44 x_{13} & -1.85 x_{3} & +  0.03 x_{25} & -1.05 x_{11} & -0.45 x_{2}\\
 x_{6}   &  1.35638594775 & -0.08 x_{18} & +  0.06 x_{26} & +  0.11 x_{28} & -1.67 x_{7} & +  0.00 x_{32} & -0.11 x_{30} & +  0.10 x_{27} & -1.69 x_{4} & -0.60 x_{8} & +  1.10 x_{10} & -1.38 x_{17} & -0.38 x_{12} & -2.68 x_{13} & -1.33 x_{3} & +  0.13 x_{25} & -1.47 x_{11} & -1.31 x_{2}\\
 x_{24}   &  30.2748275364 & -3.50 x_{18} & +  1.77 x_{26} & +  4.10 x_{28} & -45.71 x_{7} & +  0.44 x_{32} & -0.42 x_{30} & +  3.12 x_{27} & -56.83 x_{4} & -57.14 x_{8} & -24.24 x_{10} & -43.67 x_{17} & -32.35 x_{12} & -69.76 x_{13} & -3.82 x_{3} & +  0.52 x_{25} & -37.37 x_{11} & -32.19 x_{2}\\
 x_{29}   &  14.4445487284 & -0.54 x_{18} & +  1.21 x_{26} & +  1.48 x_{28} & -21.83 x_{7} & +  0.14 x_{32} & -0.89 x_{30} & +  1.39 x_{27} & -27.10 x_{4} & -6.98 x_{8} & +  3.19 x_{10} & -8.63 x_{17} & -11.20 x_{12} & -23.31 x_{13} & -28.94 x_{3} & +  1.15 x_{25} & -21.51 x_{11} & -22.83 x_{2}\\
 x_{20}   &  35.0436735336 & -3.41 x_{18} & +  2.51 x_{26} & +  3.76 x_{28} & -42.13 x_{7} & +  0.26 x_{32} & -2.22 x_{30} & +  2.69 x_{27} & -52.80 x_{4} & -36.28 x_{8} & -4.29 x_{10} & -47.60 x_{17} & -24.99 x_{12} & -67.86 x_{13} & -30.15 x_{3} & +  1.95 x_{25} & -43.10 x_{11} & -25.67 x_{2}\\
 x_{31}   &  13.4876651631 & -0.40 x_{18} & +  0.62 x_{26} & +  2.30 x_{28} & -13.04 x_{7} & +  0.86 x_{32} & +  0.07 x_{30} & +  1.37 x_{27} & -27.38 x_{4} & -18.90 x_{8} & +  6.82 x_{10} & -21.64 x_{17} & +  9.75 x_{12} & -20.56 x_{13} & -14.55 x_{3} & +  0.74 x_{25} & -19.15 x_{11} & -33.86 x_{2}\\
 x_{9}   &  2.4176463489 & -0.23 x_{18} & +  0.20 x_{26} & +  0.31 x_{28} & -4.31 x_{7} & +  0.10 x_{32} & -0.08 x_{30} & +  0.29 x_{27} & -5.41 x_{4} & -4.48 x_{8} & -2.34 x_{10} & -2.18 x_{17} & -3.29 x_{12} & -5.49 x_{13} & -2.36 x_{3} & +  0.05 x_{25} & -3.02 x_{11} & -2.32 x_{2}\\
\hline
z    &  18.4996077679 & -1.00 x_{18} & +  1.58 x_{26} & +  2.15 x_{28} & -36.07 x_{7} & +  0.31 x_{32} & -1.32 x_{30} & +  1.66 x_{27} & -33.14 x_{4} & -18.00 x_{8} & -2.22 x_{10} & -13.92 x_{17} & -12.50 x_{12} & -39.06 x_{13} & -24.33 x_{3} & +  0.74 x_{25} & -25.98 x_{11} & -14.03 x_{2}\\
\end{array}\]


 $ x_{25} $ enters and $ x_{5} $ leaves 

 \[\begin{array}{c| c c@{\hskip 2pt} c@{\hskip 2pt} c@{\hskip 2pt} c@{\hskip 2pt} c@{\hskip 2pt} c@{\hskip 2pt} c@{\hskip 2pt} c@{\hskip 2pt} c@{\hskip 2pt} c@{\hskip 2pt} c@{\hskip 2pt} c@{\hskip 2pt} c@{\hskip 2pt} c@{\hskip 2pt} c@{\hskip 2pt} c@{\hskip 2pt} c@{\hskip 2pt} }
 x_{25}   &  149.009294944 & -23.50 x_{18} & + 20.93 x_{26} & +  0.06 x_{28} & -206.80 x_{7} & + 11.54 x_{32} & -2.35 x_{30} & -1.55 x_{27} & -279.03 x_{4} & -261.99 x_{8} & -467.12 x_{10} & -118.32 x_{17} & -342.48 x_{12} & -18.25 x_{13} & -144.02 x_{3} & -210.39 x_{5} & + 206.79 x_{11} & + 250.00 x_{2}\\
 x_{1}   &  19.6226479256 & -2.83 x_{18} & +  2.64 x_{26} & +  0.23 x_{28} & -28.49 x_{7} & +  1.36 x_{32} & -0.46 x_{30} & +  0.06 x_{27} & -36.39 x_{4} & -32.99 x_{8} & -54.04 x_{10} & -14.66 x_{17} & -41.90 x_{12} & -7.27 x_{13} & -20.48 x_{3} & -24.79 x_{5} & + 20.02 x_{11} & + 27.14 x_{2}\\
 x_{19}   &  298.123781455 & -44.10 x_{18} & + 40.13 x_{26} & +  3.11 x_{28} & -415.65 x_{7} & + 20.55 x_{32} & -6.55 x_{30} & -0.09 x_{27} & -539.55 x_{4} & -500.86 x_{8} & -832.25 x_{10} & -239.95 x_{17} & -630.47 x_{12} & -95.39 x_{13} & -300.47 x_{3} & -378.22 x_{5} & + 329.10 x_{11} & + 433.16 x_{2}\\
 x_{21}   &  25.6220811607 & -3.45 x_{18} & +  2.65 x_{26} & +  1.92 x_{28} & -31.03 x_{7} & +  2.27 x_{32} & +  0.58 x_{30} & +  1.26 x_{27} & -56.69 x_{4} & -54.31 x_{8} & -73.44 x_{10} & -37.52 x_{17} & -54.48 x_{12} & -16.47 x_{13} & -16.74 x_{3} & -27.20 x_{5} & + 23.49 x_{11} & +  0.76 x_{2}\\
 x_{22}   &  223.071525731 & -32.08 x_{18} & + 29.50 x_{26} & +  3.90 x_{28} & -323.10 x_{7} & + 15.87 x_{32} & -4.55 x_{30} & +  1.59 x_{27} & -423.73 x_{4} & -383.53 x_{8} & -608.90 x_{10} & -167.88 x_{17} & -479.76 x_{12} & -85.92 x_{13} & -233.40 x_{3} & -262.15 x_{5} & + 224.66 x_{11} & + 286.12 x_{2}\\
 x_{23}   &  92.5934028565 & -12.21 x_{18} & + 11.85 x_{26} & +  3.90 x_{28} & -130.20 x_{7} & +  6.50 x_{32} & -1.71 x_{30} & +  1.99 x_{27} & -193.62 x_{4} & -173.58 x_{8} & -237.04 x_{10} & -81.35 x_{17} & -182.27 x_{12} & -71.60 x_{13} & -105.85 x_{3} & -99.00 x_{5} & + 56.36 x_{11} & + 86.01 x_{2}\\
 x_{16}   &  16.1268419859 & -2.37 x_{18} & +  2.11 x_{26} & +  0.21 x_{28} & -22.92 x_{7} & +  1.06 x_{32} & -0.40 x_{30} & +  0.04 x_{27} & -29.35 x_{4} & -26.34 x_{8} & -43.72 x_{10} & -12.71 x_{17} & -33.91 x_{12} & -6.75 x_{13} & -15.30 x_{3} & -19.61 x_{5} & + 15.99 x_{11} & + 22.44 x_{2}\\
 x_{15}   &  20.4724552256 & -3.03 x_{18} & +  2.69 x_{26} & +  0.36 x_{28} & -28.83 x_{7} & +  1.43 x_{32} & -0.39 x_{30} & +  0.08 x_{27} & -38.58 x_{4} & -35.13 x_{8} & -56.82 x_{10} & -17.11 x_{17} & -42.86 x_{12} & -8.25 x_{13} & -19.10 x_{3} & -24.74 x_{5} & + 20.94 x_{11} & + 26.77 x_{2}\\
 x_{14}   &  5.3277034686 & -0.72 x_{18} & +  0.71 x_{26} & +  0.11 x_{28} & -7.76 x_{7} & +  0.40 x_{32} & -0.15 x_{30} & +  0.00 x_{27} & -10.52 x_{4} & -9.02 x_{8} & -14.19 x_{10} & -3.93 x_{17} & -10.59 x_{12} & -1.98 x_{13} & -6.07 x_{3} & -6.17 x_{5} & +  5.01 x_{11} & +  6.88 x_{2}\\
 x_{6}   &  20.0448877805 & -3.02 x_{18} & +  2.69 x_{26} & +  0.12 x_{28} & -27.61 x_{7} & +  1.45 x_{32} & -0.41 x_{30} & -0.10 x_{27} & -36.68 x_{4} & -33.45 x_{8} & -57.49 x_{10} & -16.22 x_{17} & -43.34 x_{12} & -4.97 x_{13} & -19.39 x_{3} & -26.39 x_{5} & + 24.46 x_{11} & + 30.05 x_{2}\\
 x_{24}   &  107.062570846 & -15.61 x_{18} & + 12.56 x_{26} & +  4.13 x_{28} & -152.28 x_{7} & +  6.39 x_{32} & -1.63 x_{30} & +  2.32 x_{27} & -200.62 x_{4} & -192.15 x_{8} & -264.95 x_{10} & -104.64 x_{17} & -208.83 x_{12} & -79.16 x_{13} & -78.03 x_{3} & -108.42 x_{5} & + 69.19 x_{11} & + 96.64 x_{2}\\
 x_{29}   &  185.116640218 & -27.46 x_{18} & + 25.18 x_{26} & +  1.55 x_{28} & -258.69 x_{7} & + 13.36 x_{32} & -3.58 x_{30} & -0.39 x_{27} & -346.69 x_{4} & -307.06 x_{8} & -531.83 x_{10} & -144.16 x_{17} & -403.47 x_{12} & -44.21 x_{13} & -193.89 x_{3} & -240.97 x_{5} & + 215.34 x_{11} & + 263.52 x_{2}\\
 x_{20}   &  325.805939696 & -49.27 x_{18} & + 43.36 x_{26} & +  3.88 x_{28} & -445.65 x_{7} & + 22.79 x_{32} & -6.80 x_{30} & -0.34 x_{27} & -597.27 x_{4} & -547.51 x_{8} & -915.78 x_{10} & -278.48 x_{17} & -693.27 x_{12} & -103.46 x_{13} & -311.17 x_{3} & -410.53 x_{5} & + 360.42 x_{11} & + 462.17 x_{2}\\
 x_{31}   &  123.279301746 & -17.71 x_{18} & + 16.04 x_{26} & +  2.34 x_{28} & -165.41 x_{7} & +  9.36 x_{32} & -1.66 x_{30} & +  0.23 x_{27} & -232.97 x_{4} & -211.94 x_{8} & -337.36 x_{10} & -108.83 x_{17} & -242.59 x_{12} & -34.00 x_{13} & -120.66 x_{3} & -155.02 x_{5} & + 133.22 x_{11} & + 150.35 x_{2}\\
 x_{9}   &  10.5224438903 & -1.51 x_{18} & +  1.34 x_{26} & +  0.31 x_{28} & -15.56 x_{7} & +  0.73 x_{32} & -0.20 x_{30} & +  0.20 x_{27} & -20.59 x_{4} & -18.73 x_{8} & -27.74 x_{10} & -8.61 x_{17} & -21.92 x_{12} & -6.48 x_{13} & -10.20 x_{3} & -11.44 x_{5} & +  8.23 x_{11} & + 11.27 x_{2}\\
\hline
z    &  128.565291317 & -18.35 x_{18} & + 17.04 x_{26} & +  2.19 x_{28} & -188.82 x_{7} & +  8.84 x_{32} & -3.05 x_{30} & +  0.52 x_{27} & -239.24 x_{4} & -211.52 x_{8} & -347.25 x_{10} & -101.32 x_{17} & -265.47 x_{12} & -52.54 x_{13} & -130.71 x_{3} & -155.40 x_{5} & + 126.77 x_{11} & + 170.64 x_{2}\\
\end{array}\]


 $ x_{2} $ enters and Unbounded Dictionary!


 $ x_{2} $ enters and Unbounded Dictionary!
\end{document}
